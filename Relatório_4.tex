\documentclass[]{article}
\usepackage{lmodern}
\usepackage{amssymb,amsmath}
\usepackage{ifxetex,ifluatex}
\usepackage{fixltx2e} % provides \textsubscript
\ifnum 0\ifxetex 1\fi\ifluatex 1\fi=0 % if pdftex
  \usepackage[T1]{fontenc}
  \usepackage[utf8]{inputenc}
\else % if luatex or xelatex
  \ifxetex
    \usepackage{mathspec}
  \else
    \usepackage{fontspec}
  \fi
  \defaultfontfeatures{Ligatures=TeX,Scale=MatchLowercase}
\fi
% use upquote if available, for straight quotes in verbatim environments
\IfFileExists{upquote.sty}{\usepackage{upquote}}{}
% use microtype if available
\IfFileExists{microtype.sty}{%
\usepackage{microtype}
\UseMicrotypeSet[protrusion]{basicmath} % disable protrusion for tt fonts
}{}
\usepackage[margin=1in]{geometry}
\usepackage{hyperref}
\hypersetup{unicode=true,
            pdftitle={Ciência de Dados para Todos (Data Science For All) - 2018.1 - Análise da Produção Científica e Acadêmica da Universidade de Brasília - Relatório Final da Disciplina - Departamento de Ciência da Computação da UnB},
            pdfauthor={Jorge H. C. Fernandes, Ricardo B. Sampaio, João Ribas de Moura e Jerônimo A. Filho},
            pdfborder={0 0 0},
            breaklinks=true}
\urlstyle{same}  % don't use monospace font for urls
\usepackage{color}
\usepackage{fancyvrb}
\newcommand{\VerbBar}{|}
\newcommand{\VERB}{\Verb[commandchars=\\\{\}]}
\DefineVerbatimEnvironment{Highlighting}{Verbatim}{commandchars=\\\{\}}
% Add ',fontsize=\small' for more characters per line
\usepackage{framed}
\definecolor{shadecolor}{RGB}{248,248,248}
\newenvironment{Shaded}{\begin{snugshade}}{\end{snugshade}}
\newcommand{\KeywordTok}[1]{\textcolor[rgb]{0.13,0.29,0.53}{\textbf{#1}}}
\newcommand{\DataTypeTok}[1]{\textcolor[rgb]{0.13,0.29,0.53}{#1}}
\newcommand{\DecValTok}[1]{\textcolor[rgb]{0.00,0.00,0.81}{#1}}
\newcommand{\BaseNTok}[1]{\textcolor[rgb]{0.00,0.00,0.81}{#1}}
\newcommand{\FloatTok}[1]{\textcolor[rgb]{0.00,0.00,0.81}{#1}}
\newcommand{\ConstantTok}[1]{\textcolor[rgb]{0.00,0.00,0.00}{#1}}
\newcommand{\CharTok}[1]{\textcolor[rgb]{0.31,0.60,0.02}{#1}}
\newcommand{\SpecialCharTok}[1]{\textcolor[rgb]{0.00,0.00,0.00}{#1}}
\newcommand{\StringTok}[1]{\textcolor[rgb]{0.31,0.60,0.02}{#1}}
\newcommand{\VerbatimStringTok}[1]{\textcolor[rgb]{0.31,0.60,0.02}{#1}}
\newcommand{\SpecialStringTok}[1]{\textcolor[rgb]{0.31,0.60,0.02}{#1}}
\newcommand{\ImportTok}[1]{#1}
\newcommand{\CommentTok}[1]{\textcolor[rgb]{0.56,0.35,0.01}{\textit{#1}}}
\newcommand{\DocumentationTok}[1]{\textcolor[rgb]{0.56,0.35,0.01}{\textbf{\textit{#1}}}}
\newcommand{\AnnotationTok}[1]{\textcolor[rgb]{0.56,0.35,0.01}{\textbf{\textit{#1}}}}
\newcommand{\CommentVarTok}[1]{\textcolor[rgb]{0.56,0.35,0.01}{\textbf{\textit{#1}}}}
\newcommand{\OtherTok}[1]{\textcolor[rgb]{0.56,0.35,0.01}{#1}}
\newcommand{\FunctionTok}[1]{\textcolor[rgb]{0.00,0.00,0.00}{#1}}
\newcommand{\VariableTok}[1]{\textcolor[rgb]{0.00,0.00,0.00}{#1}}
\newcommand{\ControlFlowTok}[1]{\textcolor[rgb]{0.13,0.29,0.53}{\textbf{#1}}}
\newcommand{\OperatorTok}[1]{\textcolor[rgb]{0.81,0.36,0.00}{\textbf{#1}}}
\newcommand{\BuiltInTok}[1]{#1}
\newcommand{\ExtensionTok}[1]{#1}
\newcommand{\PreprocessorTok}[1]{\textcolor[rgb]{0.56,0.35,0.01}{\textit{#1}}}
\newcommand{\AttributeTok}[1]{\textcolor[rgb]{0.77,0.63,0.00}{#1}}
\newcommand{\RegionMarkerTok}[1]{#1}
\newcommand{\InformationTok}[1]{\textcolor[rgb]{0.56,0.35,0.01}{\textbf{\textit{#1}}}}
\newcommand{\WarningTok}[1]{\textcolor[rgb]{0.56,0.35,0.01}{\textbf{\textit{#1}}}}
\newcommand{\AlertTok}[1]{\textcolor[rgb]{0.94,0.16,0.16}{#1}}
\newcommand{\ErrorTok}[1]{\textcolor[rgb]{0.64,0.00,0.00}{\textbf{#1}}}
\newcommand{\NormalTok}[1]{#1}
\usepackage{longtable,booktabs}
\usepackage{graphicx,grffile}
\makeatletter
\def\maxwidth{\ifdim\Gin@nat@width>\linewidth\linewidth\else\Gin@nat@width\fi}
\def\maxheight{\ifdim\Gin@nat@height>\textheight\textheight\else\Gin@nat@height\fi}
\makeatother
% Scale images if necessary, so that they will not overflow the page
% margins by default, and it is still possible to overwrite the defaults
% using explicit options in \includegraphics[width, height, ...]{}
\setkeys{Gin}{width=\maxwidth,height=\maxheight,keepaspectratio}
\IfFileExists{parskip.sty}{%
\usepackage{parskip}
}{% else
\setlength{\parindent}{0pt}
\setlength{\parskip}{6pt plus 2pt minus 1pt}
}
\setlength{\emergencystretch}{3em}  % prevent overfull lines
\providecommand{\tightlist}{%
  \setlength{\itemsep}{0pt}\setlength{\parskip}{0pt}}
\setcounter{secnumdepth}{0}
% Redefines (sub)paragraphs to behave more like sections
\ifx\paragraph\undefined\else
\let\oldparagraph\paragraph
\renewcommand{\paragraph}[1]{\oldparagraph{#1}\mbox{}}
\fi
\ifx\subparagraph\undefined\else
\let\oldsubparagraph\subparagraph
\renewcommand{\subparagraph}[1]{\oldsubparagraph{#1}\mbox{}}
\fi

%%% Use protect on footnotes to avoid problems with footnotes in titles
\let\rmarkdownfootnote\footnote%
\def\footnote{\protect\rmarkdownfootnote}

%%% Change title format to be more compact
\usepackage{titling}

% Create subtitle command for use in maketitle
\newcommand{\subtitle}[1]{
  \posttitle{
    \begin{center}\large#1\end{center}
    }
}

\setlength{\droptitle}{-2em}
  \title{Ciência de Dados para Todos (Data Science For All) - 2018.1 - Análise
da Produção Científica e Acadêmica da Universidade de Brasília -
Relatório Final da Disciplina - Departamento de Ciência da
Computação da UnB}
  \pretitle{\vspace{\droptitle}\centering\huge}
  \posttitle{\par}
  \author{Jorge H. C. Fernandes, Ricardo B. Sampaio, João Ribas de Moura e
Jerônimo A. Filho}
  \preauthor{\centering\large\emph}
  \postauthor{\par}
  \predate{\centering\large\emph}
  \postdate{\par}
  \date{11/06/2018}


\begin{document}
\maketitle

\begin{verbatim}
## Warning: package 'tidyverse' was built under R version 3.4.4
\end{verbatim}

\begin{verbatim}
## Warning: package 'ggplot2' was built under R version 3.4.4
\end{verbatim}

\begin{verbatim}
## Warning: package 'tibble' was built under R version 3.4.4
\end{verbatim}

\begin{verbatim}
## Warning: package 'tidyr' was built under R version 3.4.4
\end{verbatim}

\begin{verbatim}
## Warning: package 'readr' was built under R version 3.4.4
\end{verbatim}

\begin{verbatim}
## Warning: package 'purrr' was built under R version 3.4.4
\end{verbatim}

\begin{verbatim}
## Warning: package 'forcats' was built under R version 3.4.4
\end{verbatim}

\section{Introdução}\label{introduaao}

Este relatório tem como foco principal a contextualização das áreas
de pesquisa relacionadas ao programa de Pós-Graduação em Matemática
da UnB, dos anos 2010 Ã~ 2017. Para tal está sendo apresentado dados de
produção científica dos docentes do programa, retirados do currículo
Lattes dos professores e do repositório \texttt{OASIS} do
\texttt{IBICT} (Instituto Brasileiro de Informação em Ciência e
Tecnologia).

O programa de pesquisa utilizado neste relatório é o de Matemática
(código: \texttt{53001010003P2}). O coordenador é o Carlos Alberto
Pereira dos Santos e as subáreas de concentração são:
\textbf{Álgebra, Análise, Geometria e Matemática Aplicada}. Mais
informações do programa podem ser acessadas pelo site
\href{https://sucupira.capes.gov.br/sucupira/public/consultas/coleta/programa/viewPrograma.jsf?popup=true\&cd_programa=53001010003P2}{sucupira}.

O programa de Pós-Graduação em Matemática da UnB, iniciado em 1962,
oferece Mestrado e Doutorado em Matemática nas subáreas de Álgebra,
Análise, Geometria e Matemática Aplicada (Probabilidade,
Física-Matemática e Computação). O corpo docente mantém um programa
ativo de pesquisa, participa regular e ativamente de forma destacada em
reuniões científicas e em corpos editoriais de revistas científicas,
além de manter intercâmbio científico com diversas instituições do
país e do exterior.

Há também um programa de mestrado profissional \texttt{PROFMAT}, que
é um programa de pós-graduação stricto sensu em Matemática,
reconhecido pelo Ministério da Educação e conduzindo ao título de
Mestre. Tem como objetivo proporcionar formação matemática
aprofundada relevante ao exercício da docência no Ensino Básico,
visando dar ao egresso a qualificação certificada para o exercício da
profissão de professor de Matemática.

O \texttt{PROFMAT} é um curso semipresencial ministrado por
Instituições de Ensino Superior associadas em uma Rede Nacional, no
âmbito do \textbf{Sistema Universidade Aberta do Brasil} (UAB). É
coordenado pela Comissão Acadêmica Nacional, que opera sob a égide da
Diretoria da \textbf{Sociedade Brasileira de Matemática} (SBM).

As Instituições de Ensino Superior que integram a Rede Nacional do
\texttt{PROFMAT} são denominadas Instituições Associadas e são
responsáveis, por intermédio das respectivas Coordenações
Acadêmicas Institucionais, por toda a gestão local do
\texttt{PROFMAT}, descritas no Regimento. Informações completas sobre
este mestrado podem ser encontradas na página do
\href{http://www.profmat-sbm.org.br}{\texttt{PROFMAT}}. Na Universidade
de Brasília este curso é ofertado no âmbito do Instituto de Ciências
Exatas, Departamento de Matemática.

O Departamento de Matemática da UnB obteve nota 7 na última
avaliação trienal da Capes. Essa avaliação de excelência é o
resultado de um longo, árduo e contínuo trabalho iniciado em 1962 por
professores/colaboradores corajosos e idealistas que acreditaram em
tornar o sonho do Departamento de Matemática da Universidade de
Brasília uma realidade. Desde então, esse ideal vem sendo fortalecido
com a chegada de novos professores/colaboradores que conduziram este
programa ao reconhecimento da sua excelência.

A metodologia para desenvolvimento do relatório é baseada no modelo de
mineração de dados denominado CRISP-DM (Chapman et al., 2000, Mariscal
et al., 2010).

\ldots{}

\section{Metodologia}\label{metodologia}

\section{CRISP-DM (Corresponderia Ã~ seção de
Metodologia)}\label{crisp-dm-corresponderia-a-seaao-de-metodologia}

Para desenvolvimento do trabalho devem ser seguidos, da forma mais
simplificada e coerente possível, as fases e atividades genéricas do
ciclo de vida de um projeto executado em aderência ao CRISP-DM. Em
outras palavras, a produção do relatorio deve seguir a metodologia
CRISP-DM.

``A widely used methodology for data mining is the CRoss-Industry
Standard Process for Data Mining (CRISP-DM) which mas initiated in 1996
(…) with the intent of providing a process that is \textbf{reliable
and repeatable} by people with little data-mining background, with a
framework within which experience can be recorded, to support the
replication of projects, to support planning and management, as well as
to demonstrate data mining as a mature discipline (…)'' {[}Sullivan,
Rob. Introduction to Data Mining for the Life Sciences. Springer Science
\& Business Media. 2012{]}

O seu trabalho deve conter uma seção metodologia, onde você faz uma
breve descrição da metodologia que seu grupo adotou para realização
do trabalho, que pode ser baseada no texto dessa seção, desde que
citado adequadamente.

\subsection{Delimitações iniciais}\label{delimitaaaes-iniciais}

Em aderência Ã~ estrutura do CRISP-DM, algumas delimitações de
contexto para o trabalho são apresentadas a seguir:

\subsubsection{Domínio de Aplicação do
projeto}\label{domanio-de-aplicaaao-do-projeto}

O domínio de aplicação do projeto é o da produção científica e
acadêmica brasileira, mais específicamente a produção científica ou
produção acadêmica de um subgrupo de pesquisadores vinculados Ã~
Universidade de Brasília. O domínio de aplicação do projeto deve ser
declarado na introdução ao relatório.

\subsubsection{Tipo de Problema
abordado}\label{tipo-de-problema-abordado}

O tipo de problema abordado é o da produção de análises descritivas,
quantitativas e de modelagem computacional ou estatística, que permitam
caracterizar como e porque ocorre a produção científica e acadêmica
de um grupo de pesquisadores. Essa caracterização visa subsidiar a
tomada de decisão por membros do Sistema Nacional de Pós-Graduação.
O tipo de problema abordado no projeto deve ser declarado na
introdução ao relatório.

\subsubsection{Conjunto de Ferramentas e
Técnicas}\label{conjunto-de-ferramentas-e-tacnicas}

O conjunto de requisitos relativos a ferramentas e técnicas a serem
empregadas na execução do trabalho é:

\begin{itemize}
\tightlist
\item
  O relatório deve ser entregue no formato R Markdown, apto Ã~
  geração de saída \LaTeX e PDF, composto por comandos em R
  entremeados por descrições textuais que auxiliem na interpretação
  dos resultados, bem como na compreensão do domínio de conhecimento
  sob análise.
\item
  As análises descritivas devem empregar de forma criativa as funções
  das bibliotecas de ciência de dados em R propostas por Wickham e
  Grolemund (2016).
\item
  As análises quantitativas devem lançar mão de recursos gráficos
  variados, que complementarão análises descritivas com
  \emph{insights} sobre de que forma os processos de produção
  científica e acadêmica contribuem para os resultados apresentados.
  Por exemplo, os dados analisados possibilitam justificar o eventual
  crescimento ou decréscimo de índices de produção observados?
\item
  A modelagem computacional ou estatística avançada dos dados deve
  usar uma das quatro técnicas prescritas:

  \begin{itemize}
  \tightlist
  \item
    Aprendizado de Máquina (Datacamp, 2018; Kuhn et al., 2018; Bruce e
    Bruce, 2017);
  \item
    Aprendizado Estatístico;
  \item
    Mineração de Texto ou;
  \item
    Análise de Redes (Kolaczyk e Csárdi, 2014; Lusher et al., 2013; de
    Nooy et al., 2005).
  \end{itemize}
\end{itemize}

O conjunto de requisitos relativos a ferramentas e técnicas a serem
empregadas na execução do trabalhodo projeto deve ser declarado na
parte de metodologia do relatório.

\subsection{Modelo de Referência
CRISP-DM}\label{modelo-de-referancia-crisp-dm}

Miner (2012), aprofunda: “(…) In CRISP-DM, the complete life cycle
of a data mining project is represented with \textbf{six phases}:
business understanding (determining the purpose of the study), data
understanding (data exploration and understanding), data preparation,
modeling, evaluation, and deployment.(…). {[}Miner, Gary. Practical
Text Mining and Statistical Analysis for Non-structure Text Data
Applications. Academic Press, 2012.{]}

\subsubsection{Por que usar o CRISP-DM?}\label{por-que-usar-o-crisp-dm}

Imagine uma analogia entre um projeto de datamining e a preparação de
uma receita de bolo para ser usada em uma fábrica. Para iniciar a
produção, com base numa receita de comprovada eficácia (metodológica
e científica), você tem que minerar os ingredientes (dados) em um
grande supermercado (\emph{dataset}). Com os ingredientes você precisa
aplicar um método (a forma de misturá-los), colocar os ingredientes
numa determinada ordem, mexer por um certo tempo, aquecer por tantos
minutos até o bolo ficar pronto e ser aprovado em um ou mais testes de
degustação.

Tendo por objetivo fazer com que essa receita (script de mineração de
dados) possa ser executada com sucesso diversas vezes, numa fábrica,
será que outro cozinheiro (cientista) que reproduzisse a receita
(método) chegaria ao mesmo resultado? Se a metodologia (receita) já
foi bastante testada, então é bem provável que o resultado será o
mesmo e seu produto (receita de bolo) será aceito para a produção
(\emph{deployment}) de análises para consumo futuro, com base em
fundamentos científicos.

\subsubsection{Organização hierárquica de atividades em
fases}\label{organizaaao-hierarquica-de-atividades-em-fases}

Dentro de cada fase no CRISP-DM existe uma estrutura hierárquica de
atividades genéricas para serem realizadas. Cada uma dessas atividades
\textbf{genéricas} pode determinar a execução de atividades
\textbf{específicas}.

Voltando ao exemplo do bolo, a atividade ” 1. Entendimento do Bolo”
poderia conter uma atividade genérica chamada “1.1. Determinar para
que o bolo servirá (simples café da manhã? bolo de aniversário? bolo
de casamento?)“. Dentro dessa atividade genérica poderia haver
atividades específicas como “1.1.1.Entrevistar o contratante para
obter detalhes de onde o bolo será usado?“; “1.1.2. Conversar com
os convidados sob alguma necessidade especial (sem lactose? sem
glútem?)“, etc.

\subsubsection{Seis Fases do CRISP-DM}\label{seis-fases-do-crisp-dm}

Com base no apresentado, segue uma descrição um pouco mais detalhada
das seis fases de um projeto no CRISP-DM, interpretadas no contexto
deste relatório.

\paragraph{\texorpdfstring{\textbf{Entendimento do
Negócio}}{Entendimento do Negócio}}\label{entendimento-do-negacio}

É o desenvolvimento dos objetivos e declaração das necessidades do
projeto sob a perspectiva do negócio, para transformar isso tudo em
definição de um problema de data mining. No caso desse relatório irá
se buscar encontrar dados referentes aos perfis, publicações e
orientações dos discentes dos programas de pós-graduação em
matemática, \ldots{}, . Deve-se também definir um grupo de objetivos,
um plano para a realização do projeto e um critério de sucesso. Como
objetivo para a parte de datamining desse relatório tem-se conseguir
dados referentes Ã~s seguintes áreas:

\begin{itemize}
\tightlist
\item
  Áreas com resultados mais relevantes / expressivos;\\
\item
  Nível de Internacionalização;
\item
  Principais congressos e eventos da área no Brasil e no mundo;
\item
  Principais revistas e local de publicação de trabalhos no Brasil e
  no mundo;
\item
  Quantitativo de pessoas e programas dentro do universo;
\item
  Quantitativo de orientandos atuais e que já finalizaram;
\item
  Grupos de pesquisa associados ou que os professores / pesquisadores
  fazem parte.
\end{itemize}

Quanto ao plano para realização do projeto, será realizada as
seguintes etapas:

\begin{itemize}
\tightlist
\item
  Tratamento dos dados para transformá-lo em um dataset;
\item
  Análise dos dados para a criação de resultados relevantes;
\item
  Análise dos resultados para encontrar os dados citados nos objetivos.
\end{itemize}

\paragraph{\texorpdfstring{\textbf{Entendimento dos
Dados}}{Entendimento dos Dados}}\label{entendimento-dos-dados}

O segundo estágio do CRISP-DM requer que se tenha acesso ao dado
listado nos recursos do projeto. Essa fase requer a realização da
descrição, exploração, verificação da qualidade e relatório da
qualidade dos dados.

Os dados analisados nesse relatório foram pegos a partir do arquivo
Matematica.profile.json que representa o dataset do programa de pós
graduação em Matemática (código: \texttt{53001010003P2}) e conta com
uma lista de docentes e seus dados pessoais. Essa descrição e
exploração foi feita utilizando do comando \textbf{View} no rstudio. A
qualidade dos dados se encontra bastante satisfatória, com alguns
problemas na padronização de alguns dados como o endereço que em
alguns casos apresenta diferentes nomes para um mesmo curso como
``Matematica'' e ``Matemática''.

\paragraph{\texorpdfstring{\textbf{Preparação dos
Dados}}{Preparação dos Dados}}\label{preparaaao-dos-dados}

Aqui os dados são “filtrados” retirando-se partes que não
interessam e selecionando-se os “campos" necessários para o trabalho
de mineração.

São 5 as atividades genéricas nesta fase de preparação dos dados:

\begin{itemize}
\item
  Seleção dos dados. Envolve identificar quais dados, da nossa
  ``montanha de dados'', serão realmente utilizados. Quais variáveis
  dos dados brutos serão convertidas para o \emph{dataset}? Não é
  raro cometer o erro de selecionar dados para um modelo preditivo com
  base em uma falsa ideia de que aqueles dados contém a resposta para o
  modelo que se quer construir. Surge o cuidado de se separar o sinal do
  ruído (Silver, Nate. The Signal and the Noise: Why so many
  predictions fail — but some don’t. USA: The Penguin Press HC,
  2012.).
\item
  Limpeza dos dados.
\item
  Construção dos dados. Envolve a criação de novas variáveis a
  partir de outras presentes nos \emph{datasets}.
\item
  Integração dos dados. Envolve a união (merge) de diferentes tabelas
  para criar um único \emph{dataset} para ser utilizado no R, por
  exemplo.
\item
  Formatação dos dados. Envolve a realização de pequenas
  alterações na estrutura dos dados, como a ordem das variáveis, para
  permitir a execução de determinado método de data mining.
\end{itemize}

\paragraph{\texorpdfstring{\textbf{Modelagem}}{Modelagem}}\label{modelagem}

No CRISP-DM este estágio envolve a construção e avaliação do
modelo, podendo ser realizada em quatro atividades genéricas:

\begin{itemize}
\item
  Seleção das técnicas de modelagem.
\item
  Realização de testes de modelagem, onde diferentes modelos são
  previamente testados e avaliados. Pode-se dividir o \emph{dataset}
  criado na etapa anterior para se ter uma base de treino na
  construção de modelos, e outra pequena parte para validar e avaliar
  a eficiência de cada modelo criado até se chegar ao mais
  “eficiente”.
\item
  Construção do modelo definitivo, com base na melhor experiência do
  passo anterior.
\item
  Avaliação do modelo.
\end{itemize}

\paragraph{\texorpdfstring{\textbf{Avaliação}}{Avaliação}}\label{avaliaaao}

Aqui os resultados não são apenas avaliados, mas se verifica se
existem questões relacionadas Ã~ organização que não foram
suficientemente abordadas. Deve-se refletir se o uso arepetido do modelo
criado pode trazer algum “efeito colateral” para a organização.

Nesta fase, pode-se trabalhar com 3 atividades genéricas:

\begin{itemize}
\item
  Avaliação dos resultados
\item
  Revisão do processo, por meio da qual verifica-se se o modelo foi
  construído adequadamente. As variáveis (passadas) para construir o
  modelo estarão disponíveis no futuro?
\item
  Determinação dos etapas seguintes. Pode ser necessário decidir-se
  por finalizar o projeto, passar Ã~ etapa de desenvolvimento, ou rever
  algumas fases anteriores para a melhoria do projeto.
\end{itemize}

\paragraph{\texorpdfstring{\textbf{Implantação}
(\emph{deployment})}{Implantação (deployment)}}\label{implantaaao-deployment}

Foi realizado o planejamento de implantação dos produtos desenvolvidos
(scripts, no caso do executado nesta disciplina) para o ambiente
operacional, para seu uso repetitivo, envolvendo atividades de
monitoramento e manutenção do sistema (script) desenvolvido. A fase de
implantação concluir com a produção e apresentação do relatório
final com os resultados do projeto.

São atividades genéricas na fase de \textbf{implantação}:

\begin{itemize}
\tightlist
\item
  Planejamento da transição dos produtos;
\item
  Planejamento do monitoramento dos produtos em utilização no ambiente
  operacional;
\item
  Planejamento de manuteção a ser eventualmente efetuada no produto
  (scripts);
\item
  Produção do relatório final;
\item
  Apresentação do relatório final;
\item
  Revisão sobre a execução do projeto, com registro de lições
  aprendidas etc.
\end{itemize}

No contexto do projeto realizado no âmbito desta disciplina, a
responsabilidade por execução de todas essas atividades é dos alunos,
com exceção da apresentação do relatório final, que não será
realizada.

\section{CRISP-DM Fase 1 - Entendimento do
Negócio}\label{crisp-dm-fase-1---entendimento-do-negacio}

\subsection{O que é o Sistema Nacional de Pós-Graduação?
(Contextualização)}\label{o-que-a-o-sistema-nacional-de-pas-graduaaao-contextualizaaao}

A produção do conhecimento científico, no Brasil, é
predominantemente efetuada por meio do Sistema Nacional de
Pós-Graduação - SNPG, e mais fortemente relacionada com a formação
de doutores nesse sistema (Pátaro e Mezzomo, 2013), por meio de cursos
de pós-graduação \emph{strictu sensu}.

Fernandes e Sampaio (2017) já indicaram que a ciência é
reconhecidamente um elemento essencial para o desenvolvimento social e
econômico de qualquer nação. Assim sendo, faz-se mister aprimorar o
SNPG como forma de promoção desse crescimento, visando maximizar o
retorno decorrente do emprego dos recursos nele aplicados. A promoção
do crescimento do SNPG se dá predominantemente por meio de avaliações
regulares de seus programas de pós-graduação, sob responsabilidade da
CAPES, que realiza a cada quatro anos um complexo (Leite, 2018, p.~13) e
custoso processo de coleta de dados, análise e deliberação sobre as
pós-graduações \emph{strictu sensu}, em coerência com o estabelecido
no Plano Nacional de Pós-Graduação (PNPG) 2012-2020 (CAPES, 2010) e
nos diversos documentos que definem os critérios de organização da
pós-graduação em cada área do conhecimento (CAPES, 2018). Leite
(2018) faz uma apresentação geral de como se organizam e são
avaliadas as pós-graduações no Brasil.

O Plano Nacional de Pós-Graduação (PNPG), por outro lado, define
diretrizes estratégicas para desenvolvimento da pós-graduação
brasileira, que deve abordar prioritariamente grandes temas de interesse
nacional, tais como a redução das assimetrias de desenvolvimento entre
as regiões do Brasil, a formação de professores para a educação
básica, a formação de recursos humanos para as empresas, a resposta
aos grandes desafios brasileiros sobre Água, Energia, Transporte,
Controle de Fronteiras, Agronegócio, Amazônia, Amazônia Azul (Mar),
Saúde, Defesa, Programa Espacial, além de Justiça, Segurança
Pública, Criminologia e Desequilíbrio Regional. O PNPG também traça
as diretrizes para financiamento da pós-graduação e sua
internacionalização, apresentando conclusões e recomendações.

As avaliações do SNPG, ao atribuirem mensurações de desempenho Ã~s
diversas pós-graduações que dele fazem parte, geram incentivos e
penalidades aos programas, tendo em vista a limitada disponibilidade de
recursos para investimento em bolsas, taxas de bancada etc. Embora o
sistema seja altamente sofisticado ele é também altamente criticado
(Azevedo et al., 2016), sobretudo porque há percalços na busca por um
equilíbrio entre as diferentes concepções de finalidade da ciência.
Se de um lado a promoção do conhecimento gerado predominantemente nas
ditas ciências \emph{hard} constribui para criar fluxos econômicos
mais intensos, isso não significa que essa promoção possa ocorrer em
detrimento da menor promoção na geração de conhecimento sobre
problemas sociais, predominantemente gerado nas ditas ciências
\emph{soft}, especialmente das áreas de humanidades, sob pena de
ampliação de desigualdades (Azevedo et al., 2016).

Não há solução simples, mas postula-se, nesta disciplina, que uma
maior agilidade na avaliação e a utilização de critérios mais
objetivos, poderá facilitar a melhoria do sistema.

\subsubsection{Os Colégios, Grandes Áreas e Áreas da Pós-Graduação
Brasileira}\label{os-colagios-grandes-areas-e-areas-da-pas-graduaaao-brasileira}

A partir de 2018, as diversas áreas da pós-graduação brasileira
foram organizadas na forma de colégios, grandes áreas e áreas,
conforme apresentam as tabelas a seguir.

\paragraph{Colégio de Ciências da
vida}\label{colagio-de-ciancias-da-vida}

\begin{longtable}[]{@{}lll@{}}
\toprule
CIÊNCIAS AGRÁRIAS & CIÊNCIAS BIOLÃ``GICAS & CIÊNCIAS DA
SAÚDE\tabularnewline
\midrule
\endhead
Ciência de Alimentos & Biodiversidade & Educação
Física\tabularnewline
Ciências Agrárias I & Ciências Biológicas I &
Enfermagem\tabularnewline
Medicina Veterinária & Ciências Biológicas II &
Farmácia\tabularnewline
Zootecnia / Recursos Pesqueiros & Ciências Biológicas III & Medicina
I\tabularnewline
- & - & Medicina II\tabularnewline
- & - & Medicina III\tabularnewline
- & - & Nutrição\tabularnewline
- & - & Odontologia\tabularnewline
- & - & Saúde Coletiva\tabularnewline
\bottomrule
\end{longtable}

\paragraph{Colégio de Ciências Exatas, Tecnológicas e
Multidisciplinar}\label{colagio-de-ciancias-exatas-tecnolagicas-e-multidisciplinar}

\begin{longtable}[]{@{}lll@{}}
\toprule
CIÊNCIAS EXATAS E DA TERRA & ENGENHARIAS &
MULTIDISCIPLINAR\tabularnewline
\midrule
\endhead
Astronomia / Física & Engenharias I & Biotecnologia\tabularnewline
Ciência da Computação & Engenharias II & Ciências
Ambientais\tabularnewline
Geociências & Engenharias III & Ensino\tabularnewline
Matemática / Probabilidade e Estatística & Engenharias IV &
Interdisciplinar\tabularnewline
Química & - & Materiais\tabularnewline
\bottomrule
\end{longtable}

\paragraph{Colégio de Humanidades}\label{colagio-de-humanidades}

\begin{longtable}[]{@{}lll@{}}
\toprule
\begin{minipage}[b]{0.03\columnwidth}\raggedright\strut
CIÊNCIAS HUMANAS\strut
\end{minipage} & \begin{minipage}[b]{0.03\columnwidth}\raggedright\strut
CIÊNCIAS SOCIAIS APLICADAS\strut
\end{minipage} & \begin{minipage}[b]{0.03\columnwidth}\raggedright\strut
LINGUÍSTICA, LETRAS E ARTES\strut
\end{minipage}\tabularnewline
\midrule
\endhead
\begin{minipage}[t]{0.03\columnwidth}\raggedright\strut
Antropol/Arqueol\strut
\end{minipage} & \begin{minipage}[t]{0.03\columnwidth}\raggedright\strut
Admin.Púb./Empr.,C.Contáb. e Tur.\strut
\end{minipage} & \begin{minipage}[t]{0.03\columnwidth}\raggedright\strut
Artes\strut
\end{minipage}\tabularnewline
\begin{minipage}[t]{0.03\columnwidth}\raggedright\strut
Ciência Pol. e Rel. Int.\strut
\end{minipage} & \begin{minipage}[t]{0.03\columnwidth}\raggedright\strut
Arquit., Urban. e Design\strut
\end{minipage} & \begin{minipage}[t]{0.03\columnwidth}\raggedright\strut
Linguística e Literatura\strut
\end{minipage}\tabularnewline
\begin{minipage}[t]{0.03\columnwidth}\raggedright\strut
Ciências da Religião e Teol.\strut
\end{minipage} & \begin{minipage}[t]{0.03\columnwidth}\raggedright\strut
Comunicação e Informação\strut
\end{minipage} & \begin{minipage}[t]{0.03\columnwidth}\raggedright\strut
-\strut
\end{minipage}\tabularnewline
\begin{minipage}[t]{0.03\columnwidth}\raggedright\strut
Educação\strut
\end{minipage} & \begin{minipage}[t]{0.03\columnwidth}\raggedright\strut
Direito\strut
\end{minipage} & \begin{minipage}[t]{0.03\columnwidth}\raggedright\strut
-\strut
\end{minipage}\tabularnewline
\begin{minipage}[t]{0.03\columnwidth}\raggedright\strut
Filosofia\strut
\end{minipage} & \begin{minipage}[t]{0.03\columnwidth}\raggedright\strut
Economia\strut
\end{minipage} & \begin{minipage}[t]{0.03\columnwidth}\raggedright\strut
-\strut
\end{minipage}\tabularnewline
\begin{minipage}[t]{0.03\columnwidth}\raggedright\strut
Geografia\strut
\end{minipage} & \begin{minipage}[t]{0.03\columnwidth}\raggedright\strut
Planej. Urbano e Reg. / Demografia\strut
\end{minipage} & \begin{minipage}[t]{0.03\columnwidth}\raggedright\strut
-\strut
\end{minipage}\tabularnewline
\begin{minipage}[t]{0.03\columnwidth}\raggedright\strut
História\strut
\end{minipage} & \begin{minipage}[t]{0.03\columnwidth}\raggedright\strut
Serviço Social\strut
\end{minipage} & \begin{minipage}[t]{0.03\columnwidth}\raggedright\strut
-\strut
\end{minipage}\tabularnewline
\begin{minipage}[t]{0.03\columnwidth}\raggedright\strut
Psicologia\strut
\end{minipage} & \begin{minipage}[t]{0.03\columnwidth}\raggedright\strut
-\strut
\end{minipage} & \begin{minipage}[t]{0.03\columnwidth}\raggedright\strut
-\strut
\end{minipage}\tabularnewline
\begin{minipage}[t]{0.03\columnwidth}\raggedright\strut
Sociologia\strut
\end{minipage} & \begin{minipage}[t]{0.03\columnwidth}\raggedright\strut
-\strut
\end{minipage} & \begin{minipage}[t]{0.03\columnwidth}\raggedright\strut
-\strut
\end{minipage}\tabularnewline
\bottomrule
\end{longtable}

Cada um desses colégios, grandes áreas e áreas de conhecimento
possuem dinâmicas próprias, e, portanto, não há um modelo universal
que se aplique a todas. Existem aspectos comuns, mas também grandes
peculiaridades, descritas parcialmente nos correspondentes documentos de
área disponíveis em CAPES (2018).

\subsection{A UnB dentro do Sistema Nacional de Pós-Graduação
(Contextualização)}\label{a-unb-dentro-do-sistema-nacional-de-pas-graduaaao-contextualizaaao}

\subsubsection{O que é a UnB?}\label{o-que-a-a-unb}

Descrição da Universidade de Brasília, com foco na sua produção
científica e acadêmica.

\subsubsection{Descrição das pós-graduações da
UnB}\label{descriaao-das-pas-graduaaaes-da-unb}

Texto a desenvolver.

\subsubsection{Outros aspectos que caracterizam a produção científica
e acadêmica da
UnB}\label{outros-aspectos-que-caracterizam-a-produaao-cientafica-e-acadamica-da-unb}

Texto a desenvolver.

\subsection{O que a Organização precisa realmente
alcançar?}\label{o-que-a-organizaaao-precisa-realmente-alcanaar}

Vários stakeholders estão envolvidos no projeto em curso, e
poderíamos considerar cada um deles como distintas organizações que
possuem interesses distintos e complementares. Elas são: * A Disciplina
Ciência de Dados para Todos 2018.1, que quer comprovar que seus alunos
dominam ferramentas e técnicas de ciência de dados, para fins de
avaliação de rendimento da disciplina. * A UnB, representada pelos
decanatos de pós-graduação (DPG) e de pesquisa e inovação (DPI),
que querem dispor de instrumentos para realização de avaliações
contínuas de suas pós-graduações. * O SNPG, que assim com o DPG e
DPI, também pode se beneficiar do uso de instrumentos para realização
de avaliações contínuas de suas pós-graduações. * Os interessados
em melhor conhecer o que é produzido pelo Sistema Nacional de
Pós-graduação, como empresas privadas, que querem desfrutar dos
benefícios gerados pela ciência brasileira.

A fim de dar maior fidelidade e homogeneidade ao exercício realizado na
disciplina, focaremos em atendimento aos interesses comuns das
organizações DPI, DPG e CAPES, que desejam dispor de instrumentos
ágeis para avaliação contínua da pós-graduação brasileira.

Com base no exposto, o objetivo do trabalho final a ser alcançado pelos
produtos d emineração de dados desenvolvido pelos alunos da disciplina
Ciência de Dados para Todos é produzir, tomando por base inicial os
dados fornecidos pelos professores responsáveis pela disciplina,
ferramentas para análise e avaliação contínuas e de baixo custo, do
desempenho de um conjunto de pós-graduações que estão vinculadas a
uma mesma subárea ou grupo de conhecimento. Cada área de
pós-graduação apresenta suas características peculiares, assim como
cada um dos programas vinculados a essas áreas. Como já informado,
características peculiares de cada programa podem ser obtidas a partir
de visita ao sítio da CAPES (2018).

\subsection{Avaliação das
Circunstâncias}\label{avaliaaao-das-circunstancias}

Este documento serve como base para a realização dos trabalhos dos
alunos. apresenta limitações no tocante Ã~ quantidade pequena de dados
que serão empregados para análises e avaliações, tendo em vista sua
finalidade maior que é a didática, de permitir aos alunos demonstrarem
a capacidade de aplicação das técnicas e ferramentas apreendidas
durante o semestre.

\subsubsection{Avaliação preliminar das pós-graduações na
UnB}\label{avaliaaao-preliminar-das-pas-graduaaaes-na-unb}

Texto a desenvolver.

\subsubsection{Avaliação preliminar da produção científica e
acadêmica da
UnB}\label{avaliaaao-preliminar-da-produaao-cientafica-e-acadamica-da-unb}

Texto a desenvolver.

\section{CRISP-DM Fase 2 - Entendimento dos
Dados}\label{crisp-dm-fase-2---entendimento-dos-dados}

Doravante, a fim de facilitar aos alunos seguirem a metodologia
CRISP-DM, os nomes das seções e subseções de texto serão prefixadas
com o número e nome da fase e atividade genérica do CRISP-DM. Fica
facultado aos grupos seguir ou não a sequência prevista, tendo em
vista que se pode retornar Ã~s fases anteriores, bem como podem haver
atividades que não foram adequadas Ã~s características do problema
específico sob análise.

\subsection{CRISP-DM Fase.Atividade 2.1 - Coleta inicial dos
dados}\label{crisp-dm-fase.atividade-2.1---coleta-inicial-dos-dados}

Todos os arquivos com dados iniciais a seguir apresentados foram
fornecidos pelos professores responsáveis pela disciplina. Os dados
foram gerados no mês de maio de 2018, e compilam informações entre os
anos de 2010 e 2017. Os arquivos estão no formato JSON, e seus
atributos iniciais e conteúdos são apresentados a seguir.

\subsubsection{Perfil profissional dos docentes vinculados Ã~s
pós-graduações}\label{perfil-profissional-dos-docentes-vinculados-as-pas-graduaaaes}

\begin{Shaded}
\begin{Highlighting}[]
\NormalTok{json.perfil <-}\StringTok{ "Matematica.profile.json"}
\KeywordTok{file.info}\NormalTok{(json.perfil)}
\end{Highlighting}
\end{Shaded}

\begin{verbatim}
##                           size isdir mode               mtime
## Matematica.profile.json 654736 FALSE  666 2018-07-07 17:52:28
##                                       ctime               atime exe
## Matematica.profile.json 2018-07-07 17:52:28 2018-07-07 17:52:28  no
\end{verbatim}

O arquivo Matematica.profile.json apresenta dados sobre o perfil de
todos os docentes vinculados a programas de pós-graduação da UnB,
entre 2010 e 2017. Esse arquivo foi fornecido pelos docentes
responsáveis pela disciplina.

\subsubsection{Orientações de mestrado e doutorado realizadas pelos
docentes vinculados Ã~s
pós-graduações}\label{orientaaaes-de-mestrado-e-doutorado-realizadas-pelos-docentes-vinculados-as-pas-graduaaaes}

\begin{Shaded}
\begin{Highlighting}[]
\NormalTok{json.advise <-}\StringTok{ "Matematica.advise.json"}
\KeywordTok{file.info}\NormalTok{(json.advise)}
\end{Highlighting}
\end{Shaded}

\begin{verbatim}
##                          size isdir mode               mtime
## Matematica.advise.json 235679 FALSE  666 2018-07-07 17:52:28
##                                      ctime               atime exe
## Matematica.advise.json 2018-07-07 17:52:28 2018-07-07 17:52:28  no
\end{verbatim}

O arquivo Matematica.advise.json apresenta dados sobre o orientações
de mestrado e doutorado feitas por todos os docentes vinculados a
programas de pós-graduação da UnB, entre 2010 e 2017. Esse arquivo
foi fornecido pelos docentes responsáveis pela disciplina.

\subsubsection{Produção bibliográfica gerada pelos docentes
vinculados Ã~s
pós-graduações}\label{produaao-bibliografica-gerada-pelos-docentes-vinculados-as-pas-graduaaaes}

\begin{Shaded}
\begin{Highlighting}[]
\NormalTok{json.producao.bibliografica <-}\StringTok{ "Matematica.publication.json"}
\KeywordTok{file.info}\NormalTok{(json.producao.bibliografica) }
\end{Highlighting}
\end{Shaded}

\begin{verbatim}
##                               size isdir mode               mtime
## Matematica.publication.json 313273 FALSE  666 2018-07-07 17:52:28
##                                           ctime               atime exe
## Matematica.publication.json 2018-07-07 17:52:28 2018-07-07 17:52:28  no
\end{verbatim}

O arquivo Matematica.publication.json apresenta dados sobre a produção
bibliográfica gerada por todos os docentes vinculados a programas de
pós-graduação da UnB, entre 2010 e 2017.

\subsubsection{Agrupamento dos docentes conforme áreas de
atuação}\label{agrupamento-dos-docentes-conforme-areas-de-atuaaao}

\begin{Shaded}
\begin{Highlighting}[]
\NormalTok{json.researchers_by_area <-}\StringTok{ "Matematica.researchers_by_area.json"} 
\KeywordTok{file.info}\NormalTok{(json.researchers_by_area)}
\end{Highlighting}
\end{Shaded}

\begin{verbatim}
##                                     size isdir mode               mtime
## Matematica.researchers_by_area.json 1161 FALSE  666 2018-07-07 17:52:28
##                                                   ctime
## Matematica.researchers_by_area.json 2018-07-07 17:52:28
##                                                   atime exe
## Matematica.researchers_by_area.json 2018-07-07 17:52:28  no
\end{verbatim}

O arquivo Matematica.researchers\_by\_area.json apresenta as
vinculações de todos os docentes que declararam atuar em cada uma das
áreas de pós-graduação do Sistema Nacional de Pós-Graduação da
CAPES, conforme apresenta-se registrada essa informação no currículo
Lattes de cada um, em data recente.

\begin{Shaded}
\begin{Highlighting}[]
\KeywordTok{file.info}\NormalTok{(}\StringTok{'Matematica.graph.json'}\NormalTok{)}
\end{Highlighting}
\end{Shaded}

\begin{verbatim}
##                       size isdir mode               mtime
## Matematica.graph.json 6179 FALSE  666 2018-07-07 17:52:28
##                                     ctime               atime exe
## Matematica.graph.json 2018-07-07 17:52:28 2018-07-07 17:52:28  no
\end{verbatim}

\subsubsection{Redes de colaboração entre
docentes}\label{redes-de-colaboraaao-entre-docentes}

O arquivo data/unbpos.graph.json apresenta redes de colaboração na
co-autoria de artigos cientpificos, feitas entre os docentes vinculados
a programas de pós-graduação da UnB, entre 2010 e 2017.

\subsection{CRISP-DM Fase.Atividade 2.2 - Descrição dos
Dados}\label{crisp-dm-fase.atividade-2.2---descriaao-dos-dados}

Para ler e manipular inicialmente esses dados, serão usadas
primordialmente as bibliotecas seguintes

\begin{Shaded}
\begin{Highlighting}[]
\KeywordTok{library}\NormalTok{(jsonlite)}
\end{Highlighting}
\end{Shaded}

\begin{verbatim}
## 
## Attaching package: 'jsonlite'
\end{verbatim}

\begin{verbatim}
## The following object is masked from 'package:purrr':
## 
##     flatten
\end{verbatim}

\begin{Shaded}
\begin{Highlighting}[]
\KeywordTok{library}\NormalTok{(listviewer)}
\end{Highlighting}
\end{Shaded}

\begin{verbatim}
## Warning: package 'listviewer' was built under R version 3.4.4
\end{verbatim}

\begin{Shaded}
\begin{Highlighting}[]
\KeywordTok{library}\NormalTok{(readxl)}
\end{Highlighting}
\end{Shaded}

\begin{verbatim}
## Warning: package 'readxl' was built under R version 3.4.4
\end{verbatim}

\begin{Shaded}
\begin{Highlighting}[]
\KeywordTok{library}\NormalTok{(readr)}
\CommentTok{#library(readtext)}
\KeywordTok{library}\NormalTok{(ggplot2)}
\KeywordTok{suppressMessages}\NormalTok{(}\KeywordTok{library}\NormalTok{(}\StringTok{"tidyverse"}\NormalTok{))}
\end{Highlighting}
\end{Shaded}

Como já informado, a descrição dos dados verifica se os dados sendo
acessados terão potencial para responder Ã~s questões de \emph{data
mining}. Além disso, deve-se avaliar qual o volume de dados, a
estrutura dos dados (tipos), codificações usadas, etc. Neste projeto,
a descrição dos dados é responsabilidade parcial dos alunos, tendo em
vista que esta seção já oferece uma descrição inicial simplificada.
O relatório final deve conter descrições significativas e
aprofundadas dos dados.

\subsubsection{Descrição dos dados do
perfil}\label{descriaao-dos-dados-do-perfil}

O arquivo unb.perfis.json, que contém dados que caracterizam o perfil
profissional de todos os docentes do grupo sob análise, podem ser lido
por meio do comando seguinte.

\begin{Shaded}
\begin{Highlighting}[]
\NormalTok{unb.prof <-}\StringTok{ }\KeywordTok{fromJSON}\NormalTok{(}\StringTok{"Matematica.profile.json"}\NormalTok{)}
\end{Highlighting}
\end{Shaded}

A quantidade de docentes sob análise é apresentada a seguir.

\begin{Shaded}
\begin{Highlighting}[]
\KeywordTok{length}\NormalTok{(unb.prof)}
\end{Highlighting}
\end{Shaded}

\begin{verbatim}
## [1] 44
\end{verbatim}

Para gerar uma apresentação inicial dos dados que estão contido nos
dados de perfil dos docentes, pode-se usar a função glimpse, da
biblioteca dplyr, como ilustra o código seguinte, que apresenta os
atributos típicos que podem ser obtidos relativamente a um pesquisador
específico, o mais antigo docente ainda em exercício na UnB a ter
criado seu registro na plataforma Lattes.

\begin{Shaded}
\begin{Highlighting}[]
\KeywordTok{library}\NormalTok{(dplyr) }
\KeywordTok{glimpse}\NormalTok{(unb.prof[[}\DecValTok{1}\NormalTok{]], }\DataTypeTok{width =} \DecValTok{30}\NormalTok{)}
\end{Highlighting}
\end{Shaded}

\begin{verbatim}
## List of 7
##  $ nome                  : chr "Norai Romeu Rocco"
##  $ resumo_cv             : chr "Possui graduação em Matemática (licenciatura plena) pela Universidade Estadual Paulista Júlio de Mesquita Filho"| __truncated__
##  $ areas_de_atuacao      :'data.frame':  5 obs. of  4 variables:
##   ..$ grande_area  : chr [1:5] "CIENCIAS_EXATAS_E_DA_TERRA" "CIENCIAS_EXATAS_E_DA_TERRA" "CIENCIAS_EXATAS_E_DA_TERRA" "CIENCIAS_EXATAS_E_DA_TERRA" ...
##   ..$ area         : chr [1:5] "Matemática" "Matemática" "Matemática" "Ciência da Computação" ...
##   ..$ sub_area     : chr [1:5] "" "Álgebra" "Álgebra" "Matemática da Computação" ...
##   ..$ especialidade: chr [1:5] "" "" "Grupos de Álgebra Não-Comutaviva" "Matemática Simbólica" ...
##  $ endereco_profissional :List of 8
##   ..$ instituicao: chr "Universidade de Brasília"
##   ..$ orgao      : chr "Instituto de Ciências Exatas"
##   ..$ unidade    : chr "Departamento de Matemática"
##   ..$ DDD        : chr "061"
##   ..$ telefone   : chr "31076442"
##   ..$ bairro     : chr "Asa Norte"
##   ..$ cep        : chr "70910900"
##   ..$ cidade     : chr "Brasília"
##  $ producao_bibiografica :List of 4
##   ..$ ARTIGO_ACEITO                         :'data.frame':   1 obs. of  10 variables:
##   .. ..$ natureza        : chr "NAO_INFORMADO"
##   .. ..$ titulo          : chr "Finiteness conditions for the non-abelian tensor product of groups"
##   .. ..$ periodico       : chr "MONATSHEFTE FUR MATHEMATIK"
##   .. ..$ ano             : chr "2017"
##   .. ..$ volume          : chr ""
##   .. ..$ issn            : chr "00269255"
##   .. ..$ paginas         : chr " - "
##   .. ..$ doi             : chr "10.1007/s00605-017-1143-x"
##   .. ..$ autores         :List of 1
##   .. ..$ autores-endogeno:List of 1
##   ..$ DEMAIS_TIPOS_DE_PRODUCAO_BIBLIOGRAFICA:'data.frame':   7 obs. of  9 variables:
##   .. ..$ natureza          : chr [1:7] "DIVULGAÇÃO DE RESULTADOS DE PESQUISA" "DIVULGAÇÃO DE RESULTADOS DE PESQUISA" "DIVULGAÇÃO DE RESULTADOS DE PESQUISA" "DIVULGAÇÃO DE RESULTADOS DE PESQUISA" ...
##   .. ..$ titulo            : chr [1:7] "NON-ABELIAN TENSOR SQUARE OF FINITE-BY-NILPOTENT GROUPS" "The q-tensor square of finitely generated nilpotent groups, q >=0" "The q-tensor square of finitely generated nilpotent groups, q >=0" "THE NON-ABELIAN TENSOR SQUARE OF RESIDUALLY FINITE GROUPS" ...
##   .. ..$ ano               : chr [1:7] "2015" "2016" "2016" "2016" ...
##   .. ..$ pais_de_publicacao: chr [1:7] "Estados Unidos" "Estados Unidos" "Estados Unidos" "Estados Unidos" ...
##   .. ..$ editora           : chr [1:7] "" "ArXiv.com - Cornell University Library" "ArXiv.com - Cornell University Library" "ArXiv.com - Cornell University Library" ...
##   .. ..$ doi               : chr [1:7] "" "" "" "" ...
##   .. ..$ numero_de_paginas : chr [1:7] "8" "12" "12" "11" ...
##   .. ..$ autores           :List of 7
##   .. ..$ autores-endogeno  :List of 7
##   ..$ EVENTO                                :'data.frame':   1 obs. of  11 variables:
##   .. ..$ natureza        : chr "RESUMO"
##   .. ..$ titulo          : chr "On Semidirect Products and non-abelian Tensor Products of Groups"
##   .. ..$ nome_do_evento  : chr "XIX Colóquio Latinoamericano de Álgebra"
##   .. ..$ ano_do_trabalho : chr "2012"
##   .. ..$ pais_do_evento  : chr "Chile"
##   .. ..$ cidade_do_evento: chr "Pucón - Chile"
##   .. ..$ doi             : chr ""
##   .. ..$ classificacao   : chr "INTERNACIONAL"
##   .. ..$ paginas         : chr " - "
##   .. ..$ autores         :List of 1
##   .. ..$ autores-endogeno:List of 1
##   ..$ PERIODICO                             :'data.frame':   6 obs. of  10 variables:
##   .. ..$ natureza        : chr [1:6] "COMPLETO" "COMPLETO" "COMPLETO" "COMPLETO" ...
##   .. ..$ titulo          : chr [1:6] "On the q-tensor square of a group" "A survey of non-abelian tensor products of groups and related constructions" "The q-tensor square of finitely generated nilpotent groups, q odd" "Non-abelian tensor square of finite-by-nilpotent groups" ...
##   .. ..$ periodico       : chr [1:6] "Journal of Group Theory" "Boletim da Sociedade Paranaense de Matemática" "JOURNAL OF ALGEBRA AND ITS APPLICATIONS" "Archiv der Mathematik (Printed ed.)" ...
##   .. ..$ ano             : chr [1:6] "2011" "2012" "2016" "2016" ...
##   .. ..$ volume          : chr [1:6] "14" "30" "16" "107" ...
##   .. ..$ issn            : chr [1:6] "14335883" "21751188" "02194988" "0003889X" ...
##   .. ..$ paginas         : chr [1:6] "785 - 805" "77 - 89" "1750211 - " "127 - 133" ...
##   .. ..$ doi             : chr [1:6] "10.1515/JGT.2010.084" "10.5269/bspm.v30i1.13350" "10.1142/S0219498817502115" "10.1007/s00013-016-0930-2" ...
##   .. ..$ autores         :List of 6
##   .. ..$ autores-endogeno:List of 6
##  $ orientacoes_academicas:List of 3
##   ..$ ORIENTACAO_CONCLUIDA_DOUTORADO   :'data.frame':    3 obs. of  13 variables:
##   .. ..$ natureza                   : chr [1:3] "Tese de doutorado" "Tese de doutorado" "Tese de doutorado"
##   .. ..$ titulo                     : chr [1:3] "Cotas superiores para o expoente e o número mínimo de geradores do quadrado q-tensorial de grupos nilpotentes, q geq 0." "Uma Apresentação Policíclica para o Quadrado q-Tensorial de um Grupo Policíclico" "Quadrado Tensorial Não-Abeliano de p-Grupos Finitos com Subgrupo Derivado de Ordem p, p ímpar"
##   .. ..$ ano                        : chr [1:3] "2011" "2011" "2017"
##   .. ..$ id_lattes_aluno            : chr [1:3] "9037151037918091" "8664599889120339" "0723203301483174"
##   .. ..$ nome_aluno                 : chr [1:3] "Eunice Cândida Pereira Rodrigues" "Ivonildes Ribeiro Martins" "Cleilton Aparecido Canal"
##   .. ..$ instituicao                : chr [1:3] "Universidade de Brasília" "Universidade de Brasília" "Universidade de Brasília"
##   .. ..$ curso                      : chr [1:3] "Matemática" "Matemática" "Matemática"
##   .. ..$ codigo_do_curso            : chr [1:3] "51500035" "51500035" "51500035"
##   .. ..$ bolsa                      : chr [1:3] "SIM" "SIM" "NAO"
##   .. ..$ agencia_financiadora       : chr [1:3] "Fundação de Amparo à Pesquisa do Estado de Mato Grosso" "Conselho Nacional de Desenvolvimento Científico e Tecnológico" ""
##   .. ..$ codigo_agencia_financiadora: chr [1:3] "035600000004" "002200000000" ""
##   .. ..$ nome_orientadores          :List of 3
##   .. ..$ id_lattes_orientadores     :List of 3
##   ..$ ORIENTACAO_CONCLUIDA_MESTRADO    :'data.frame':    3 obs. of  13 variables:
##   .. ..$ natureza                   : chr [1:3] "Dissertação de mestrado" "Dissertação de mestrado" "Dissertação de mestrado"
##   .. ..$ titulo                     : chr [1:3] "Algumas Cotas Súperiores para aordem do Quadrado Tensorial não abeliano de um Grupo" "O Grau de Permutabilidade de Subgrupos de um Grupo Finito" "Sobre pE-grupos e pA-grupos finitos"
##   .. ..$ ano                        : chr [1:3] "2010" "2011" "2012"
##   .. ..$ id_lattes_aluno            : chr [1:3] "5367744818899315" "" "0121355793029434"
##   .. ..$ nome_aluno                 : chr [1:3] "Bruno Cesar Rodrigues Lima" "Mônica Aparecida Crunivel Valadão" "Marina Gabriella Ribeiro Bardella"
##   .. ..$ instituicao                : chr [1:3] "Universidade de Brasília" "Universidade de Brasília" "Universidade de Brasília"
##   .. ..$ curso                      : chr [1:3] "Matemática" "Matemática" "Matemática"
##   .. ..$ codigo_do_curso            : chr [1:3] "51500035" "51500035" "51500035"
##   .. ..$ bolsa                      : chr [1:3] "NAO" "SIM" "SIM"
##   .. ..$ agencia_financiadora       : chr [1:3] "" "Coordenação de Aperfeiçoamento de Pessoal de Nível Superior" "Coordenação de Aperfeiçoamento de Pessoal de Nível Superior"
##   .. ..$ codigo_agencia_financiadora: chr [1:3] "" "045000000000" "045000000000"
##   .. ..$ nome_orientadores          :List of 3
##   .. ..$ id_lattes_orientadores     :List of 3
##   ..$ ORIENTACAO_EM_ANDAMENTO_DOUTORADO:'data.frame':    1 obs. of  13 variables:
##   .. ..$ natureza                   : chr "Tese de doutorado"
##   .. ..$ titulo                     : chr "Quadrado Tensorial não Abeliano de certas classes de Grupos Finitos"
##   .. ..$ ano                        : chr "2014"
##   .. ..$ id_lattes_aluno            : chr "1933036212945705"
##   .. ..$ nome_aluno                 : chr "Juliana Silva Canella"
##   .. ..$ instituicao                : chr "Universidade de Brasília"
##   .. ..$ curso                      : chr "Matemática"
##   .. ..$ codigo_do_curso            : chr "51500035"
##   .. ..$ bolsa                      : chr "SIM"
##   .. ..$ agencia_financiadora       : chr "Coordenação de Aperfeiçoamento de Pessoal de Nível Superior"
##   .. ..$ codigo_agencia_financiadora: chr "045000000000"
##   .. ..$ nome_orientadores          :List of 1
##   .. ..$ id_lattes_orientadores     :List of 1
##  $ senioridade           : chr "8"
\end{verbatim}

Uma breve inspeção visual dos atributos anteriormente apresentados
permite inferir que o pesquisador sob análise:

\begin{itemize}
\tightlist
\item
  Atua predominantemente na área de matemática.
\item
  Trabalha no Instituto de Ciências Exatas da UnB.
\item
  Possui três artigos recentes publicados, além de um aceito para
  publicação.
\item
  Possui uma orientação de doutorado em andamento, iniciada em 2014.
\item
  Foi classificado com senioridade 5.
\end{itemize}

\paragraph{Potencial de utilização dos dados do perfil dos
docentes}\label{potencial-de-utilizaaao-dos-dados-do-perfil-dos-docentes}

Esses dados terão potencial para responder Ã~s questões de \emph{data
mining}? O que é possível gerar a partir desses dados, para o conjunto
dos 1592 docentes da UnB? A fim de compreender a relevância dos dados
para a avaliação da produção acadêmica nas pós-graduações
brasileiras pode-se recorrer a trabalhos como os seguintes:

\begin{itemize}
\tightlist
\item
  Leite (2018) apresenta, em suas ``Considerações básicas sobre a
  Avaliação do Sistema Nacional de Pós-Graduação'', o conjunto dos
  itens que são tópicos de avaliação das pós-graduações pela
  CAPES, e que envolvem, entre outros:

  \begin{itemize}
  \tightlist
  \item
    Avaliação do corpo docente, com 20\% a 30\% de peso na avaliação
    total do programa, a depender do seu tipo. Analisando-se de forma
    mais detalhada os critérios de avaliação do corpo docente,
    indicados por Leite, o que é possível gerar com base nos dados
    disponíveis em unb.prof? Há dados que permitam identificar o
    perfil do docente, como proposto pela CAPES, inclusive no documento
    de área específica na qual atua o pesquisador? Que outros aspectos
    relevantes para a CAPES podem ser levantados com base nos dados
    dessa fonte?
  \item
    Avaliação do corpo discente, Teses e dissertações, com 30\% a
    20\% de peso na avaliação total do programa, a depender de seu
    tipo. Os dados sobre orientação permitem fazer quais tipos de
    avaliações do corpo discente?
  \item
    Avaliação da produção intelectual, com 40\% de peso na
    avaliação total. Qual a relevância dos dados em unb.prof para
    essa avaliação? Que outros arquivos podem melhor subsidiar essa
    avaliação?
  \end{itemize}
\item
  Em busca de considerar outros fatores relevantes para a avaliação da
  pós-graduação, não considerados no modelo da CAPES, pode-se
  recorrer ao trabalho de Kalpazidou Schmidt e Graversen (2018), que
  apresentam um conjunto de fatores persistentes que facilitam a
  existência de ambientes de pesquisa inovadores e dinâmicos, dentre
  os quais se destaca:

  \begin{itemize}
  \tightlist
  \item
    Atividade em pesquisas com elevado grau de impacto social;
  \item
    Promoção de elevado grau de autonomia individual, tanto do ponto
    de vista teórico quanto metodológico;
  \item
    Possuem um bom clima de trabalho, baseado no trabalho em times;
  \item
    São internacioinalmente bem conhecidas etc.
  \end{itemize}

  Estariam esses fatores contemplados, de alguma forma, memso que
  parcialmente, nos dados presentes em unb.prof? Ou em qualquer outros
  dos arquivos? Cabe explorar.
\end{itemize}

\subsubsection{Descrição dos dados de
orientações}\label{descriaao-dos-dados-de-orientaaaes}

\begin{Shaded}
\begin{Highlighting}[]
\NormalTok{unb.adv <-}\StringTok{ }\KeywordTok{fromJSON}\NormalTok{(}\StringTok{"Matematica.advise.json"}\NormalTok{)}
\KeywordTok{names}\NormalTok{(unb.adv)}
\end{Highlighting}
\end{Shaded}

\begin{verbatim}
## [1] "ORIENTACAO_EM_ANDAMENTO_DE_POS_DOUTORADO"    
## [2] "ORIENTACAO_EM_ANDAMENTO_DOUTORADO"           
## [3] "ORIENTACAO_EM_ANDAMENTO_MESTRADO"            
## [4] "ORIENTACAO_EM_ANDAMENTO_GRADUACAO"           
## [5] "ORIENTACAO_EM_ANDAMENTO_INICIACAO_CIENTIFICA"
## [6] "ORIENTACAO_CONCLUIDA_POS_DOUTORADO"          
## [7] "ORIENTACAO_CONCLUIDA_DOUTORADO"              
## [8] "ORIENTACAO_CONCLUIDA_MESTRADO"               
## [9] "OUTRAS_ORIENTACOES_CONCLUIDAS"
\end{verbatim}

\begin{Shaded}
\begin{Highlighting}[]
\KeywordTok{names}\NormalTok{(unb.adv}\OperatorTok{$}\NormalTok{ORIENTACAO_CONCLUIDA_DOUTORADO)}
\end{Highlighting}
\end{Shaded}

\begin{verbatim}
## [1] "2010" "2011" "2012" "2013" "2014" "2015" "2016" "2017"
\end{verbatim}

\begin{Shaded}
\begin{Highlighting}[]
\KeywordTok{length}\NormalTok{(unb.adv}\OperatorTok{$}\NormalTok{ORIENTACAO_CONCLUIDA_DOUTORADO}\OperatorTok{$}\StringTok{`}\DataTypeTok{2016}\StringTok{`}\OperatorTok{$}\NormalTok{natureza)}
\end{Highlighting}
\end{Shaded}

\begin{verbatim}
## [1] 16
\end{verbatim}

\begin{Shaded}
\begin{Highlighting}[]
\KeywordTok{head}\NormalTok{(}\KeywordTok{sort}\NormalTok{(}\KeywordTok{table}\NormalTok{(unb.adv}\OperatorTok{$}\NormalTok{ORIENTACAO_CONCLUIDA_DOUTORADO}\OperatorTok{$}\StringTok{`}\DataTypeTok{2017}\StringTok{`}\OperatorTok{$}\NormalTok{curso), }\DataTypeTok{decreasing =} \OtherTok{TRUE}\NormalTok{), }\DecValTok{10}\NormalTok{)}
\end{Highlighting}
\end{Shaded}

\begin{verbatim}
## 
##              Matemática Doutorado em Matematica Doutorado em Matemática 
##                       8                       2                       1 
##             Informática              Matematica 
##                       1                       1
\end{verbatim}

\begin{Shaded}
\begin{Highlighting}[]
\KeywordTok{head}\NormalTok{(}\KeywordTok{sort}\NormalTok{(}\KeywordTok{table}\NormalTok{(unb.adv}\OperatorTok{$}\NormalTok{ORIENTACAO_CONCLUIDA_MESTRADO}\OperatorTok{$}\StringTok{`}\DataTypeTok{2017}\StringTok{`}\OperatorTok{$}\NormalTok{curso), }\DataTypeTok{decreasing =} \OtherTok{TRUE}\NormalTok{), }\DecValTok{10}\NormalTok{)}
\end{Highlighting}
\end{Shaded}

\begin{verbatim}
## 
##                           Matemática Mestrado em Matemática e Estatística 
##                                   13                                    2 
##                          Informática                           Matematica 
##                                    1                                    1
\end{verbatim}

\subsubsection{Descrição dos dados de produção
bibliográfica}\label{descriaao-dos-dados-de-produaao-bibliografica}

\begin{Shaded}
\begin{Highlighting}[]
\NormalTok{unb.pub <-}\StringTok{ }\KeywordTok{fromJSON}\NormalTok{(}\StringTok{"Matematica.publication.json"}\NormalTok{)}
\KeywordTok{names}\NormalTok{(unb.pub)}
\end{Highlighting}
\end{Shaded}

\begin{verbatim}
## [1] "PERIODICO"                             
## [2] "LIVRO"                                 
## [3] "CAPITULO_DE_LIVRO"                     
## [4] "TEXTO_EM_JORNAIS"                      
## [5] "EVENTO"                                
## [6] "ARTIGO_ACEITO"                         
## [7] "DEMAIS_TIPOS_DE_PRODUCAO_BIBLIOGRAFICA"
\end{verbatim}

\begin{Shaded}
\begin{Highlighting}[]
\KeywordTok{names}\NormalTok{(unb.pub}\OperatorTok{$}\NormalTok{PERIODICO}\OperatorTok{$}\StringTok{`}\DataTypeTok{2012}\StringTok{`}\NormalTok{)}
\end{Highlighting}
\end{Shaded}

\begin{verbatim}
##  [1] "natureza"         "titulo"           "periodico"       
##  [4] "ano"              "volume"           "issn"            
##  [7] "paginas"          "doi"              "autores"         
## [10] "autores-endogeno"
\end{verbatim}

\begin{Shaded}
\begin{Highlighting}[]
\KeywordTok{head}\NormalTok{(}\KeywordTok{sort}\NormalTok{(}\KeywordTok{table}\NormalTok{(unb.pub}\OperatorTok{$}\NormalTok{PERIODICO}\OperatorTok{$}\StringTok{`}\DataTypeTok{2017}\StringTok{`}\OperatorTok{$}\NormalTok{periodico), }\DataTypeTok{decreasing =} \OtherTok{TRUE}\NormalTok{), }\DecValTok{10}\NormalTok{)}
\end{Highlighting}
\end{Shaded}

\begin{verbatim}
## 
##                   ANNALI DI MATEMATICA PURA ED APPLICATA 
##                                                        4 
##          Bulletin of the Australian Mathematical Society 
##                                                        4 
##                               Journal of Algebra (Print) 
##                                                        4 
##                                       JOURNAL OF ALGEBRA 
##                                                        3 
##             Electronic Journal of Differential Equations 
##                                                        2 
##                            ISRAEL JOURNAL OF MATHEMATICS 
##                                                        2 
##           Journal of Fixed Point Theory and Applications 
##                                                        2 
##                          JOURNAL OF GEOMETRY AND PHYSICS 
##                                                        2 
##                                  JOURNAL OF GROUP THEORY 
##                                                        2 
## JOURNAL OF THE LONDON MATHEMATICAL SOCIETY-SECOND SERIES 
##                                                        2
\end{verbatim}

\begin{Shaded}
\begin{Highlighting}[]
\KeywordTok{head}\NormalTok{(}\KeywordTok{sort}\NormalTok{(}\KeywordTok{table}\NormalTok{(unb.pub}\OperatorTok{$}\NormalTok{LIVRO}\OperatorTok{$}\StringTok{`}\DataTypeTok{2015}\StringTok{`}\OperatorTok{$}\NormalTok{nome_da_editora), }\DataTypeTok{decreasing =} \OtherTok{TRUE}\NormalTok{), }\DecValTok{10}\NormalTok{)}
\end{Highlighting}
\end{Shaded}

\begin{verbatim}
## Elsevier ENTCS - Electronic Notes in Theoretical Computer Science 
##                                                                 1
\end{verbatim}

\subsubsection{Descrição dos dados de agregação de docentes por
área}\label{descriaao-dos-dados-de-agregaaao-de-docentes-por-area}

\begin{Shaded}
\begin{Highlighting}[]
\NormalTok{unb.area <-}\StringTok{ }\KeywordTok{fromJSON}\NormalTok{(}\StringTok{"Matematica.researchers_by_area.json"}\NormalTok{)}
\NormalTok{unb.area.df <-}\StringTok{ }\KeywordTok{cbind}\NormalTok{(}\KeywordTok{names}\NormalTok{(unb.area}\OperatorTok{$}\StringTok{`}\DataTypeTok{Areas dos pesquisadores}\StringTok{`}\NormalTok{),}
\NormalTok{           (}\KeywordTok{sapply}\NormalTok{(unb.area}\OperatorTok{$}\StringTok{`}\DataTypeTok{Areas dos pesquisadores}\StringTok{`}\NormalTok{, }\ControlFlowTok{function}\NormalTok{(x) }\KeywordTok{length}\NormalTok{(x))))}
\KeywordTok{rownames}\NormalTok{(unb.area.df) <-}\StringTok{ }\KeywordTok{c}\NormalTok{(}\DecValTok{1}\OperatorTok{:}\KeywordTok{nrow}\NormalTok{(unb.area.df)); }\KeywordTok{colnames}\NormalTok{(unb.area.df) <-}\StringTok{ }\KeywordTok{c}\NormalTok{(}\StringTok{"Area"}\NormalTok{, }\StringTok{"Professores"}\NormalTok{)}
\KeywordTok{glimpse}\NormalTok{(unb.area.df)}
\end{Highlighting}
\end{Shaded}

\begin{verbatim}
##  chr [1:6, 1:2] "Ciência da Computação" "Engenharia Aeroespacial" ...
##  - attr(*, "dimnames")=List of 2
##   ..$ : chr [1:6] "1" "2" "3" "4" ...
##   ..$ : chr [1:2] "Area" "Professores"
\end{verbatim}

\begin{Shaded}
\begin{Highlighting}[]
\KeywordTok{head}\NormalTok{(unb.area.df[])}
\end{Highlighting}
\end{Shaded}

\begin{verbatim}
##   Area                          Professores
## 1 "Ciência da Computação"       "4"        
## 2 "Engenharia Aeroespacial"     "1"        
## 3 "Engenharia Biomédica"        "1"        
## 4 "Física"                      "1"        
## 5 "Matemática"                  "40"       
## 6 "Probabilidade e Estatística" "2"
\end{verbatim}

\subsubsection{Descrição dos dados de redes de
colaboração}\label{descriaao-dos-dados-de-redes-de-colaboraaao}

\subsection{CRISP-DM Fase.Atividade 2.3 - Análise exploratória dos
dados}\label{crisp-dm-fase.atividade-2.3---analise-explorataria-dos-dados}

Como já informado, a análise exploratória dos dados possibilita um
entendimento mais profundo da relação estatística existente entre os
dados dos \emph{datasets} para um melhor entendimento da qualidade
daqueles dados para os objetivos do projeto.

Como já informado, a análise exploratória dos dados é
responsabilidade parcial dos alunos, tendo em vista que este relatório
apresenta uma análise exploratória preliminar. O relatório final deve
conter análises exploratórias dos dados que sejam significativas e
aprofundadas.

\subsubsection{Arquivo Profile}\label{arquivo-profile}

\begin{Shaded}
\begin{Highlighting}[]
\CommentTok{# jsonedit(unb.prof)}
\CommentTok{# Número de áreas de atuação cumulativo}
\KeywordTok{sum}\NormalTok{(}\KeywordTok{sapply}\NormalTok{(unb.prof, }\ControlFlowTok{function}\NormalTok{(x) }\KeywordTok{nrow}\NormalTok{(x}\OperatorTok{$}\NormalTok{areas_de_atuacao)))}
\end{Highlighting}
\end{Shaded}

\begin{verbatim}
## [1] 104
\end{verbatim}

\begin{Shaded}
\begin{Highlighting}[]
\CommentTok{# Número de áreas de atuação por pessoa}
\KeywordTok{table}\NormalTok{(}\KeywordTok{unlist}\NormalTok{(}\KeywordTok{sapply}\NormalTok{(unb.prof, }\ControlFlowTok{function}\NormalTok{(x) }\KeywordTok{nrow}\NormalTok{(x}\OperatorTok{$}\NormalTok{areas_de_atuacao))))}
\end{Highlighting}
\end{Shaded}

\begin{verbatim}
## 
##  1  2  3  4  5  6 
## 14 12 10  5  2  1
\end{verbatim}

\begin{Shaded}
\begin{Highlighting}[]
\CommentTok{# Número de pessoas por grande area}
\KeywordTok{table}\NormalTok{(}\KeywordTok{unlist}\NormalTok{(}\KeywordTok{sapply}\NormalTok{(unb.prof, }\ControlFlowTok{function}\NormalTok{(x) (x}\OperatorTok{$}\NormalTok{areas_de_atuacao}\OperatorTok{$}\NormalTok{grande_area))))}
\end{Highlighting}
\end{Shaded}

\begin{verbatim}
## 
##                            CIENCIAS_EXATAS_E_DA_TERRA 
##                          2                        100 
##                ENGENHARIAS 
##                          2
\end{verbatim}

\begin{Shaded}
\begin{Highlighting}[]
\CommentTok{# Número de pessoas que produziram os específicos tipos de produção}
\KeywordTok{table}\NormalTok{(}\KeywordTok{unlist}\NormalTok{(}\KeywordTok{sapply}\NormalTok{(unb.prof, }\ControlFlowTok{function}\NormalTok{(x) }\KeywordTok{names}\NormalTok{(x}\OperatorTok{$}\NormalTok{producao_bibiografica))))}
\end{Highlighting}
\end{Shaded}

\begin{verbatim}
## 
##                          ARTIGO_ACEITO 
##                                      7 
##                      CAPITULO_DE_LIVRO 
##                                     11 
## DEMAIS_TIPOS_DE_PRODUCAO_BIBLIOGRAFICA 
##                                      9 
##                                 EVENTO 
##                                     20 
##                                  LIVRO 
##                                      6 
##                              PERIODICO 
##                                     44 
##                       TEXTO_EM_JORNAIS 
##                                      1
\end{verbatim}

\begin{Shaded}
\begin{Highlighting}[]
\CommentTok{# Número de publicações por tipo}
\KeywordTok{sum}\NormalTok{(}\KeywordTok{sapply}\NormalTok{(unb.prof, }\ControlFlowTok{function}\NormalTok{(x) }\KeywordTok{length}\NormalTok{(x}\OperatorTok{$}\NormalTok{producao_bibiografica}\OperatorTok{$}\NormalTok{ARTIGO_ACEITO}\OperatorTok{$}\NormalTok{ano)))}
\end{Highlighting}
\end{Shaded}

\begin{verbatim}
## [1] 7
\end{verbatim}

\begin{Shaded}
\begin{Highlighting}[]
\KeywordTok{sum}\NormalTok{(}\KeywordTok{sapply}\NormalTok{(unb.prof, }\ControlFlowTok{function}\NormalTok{(x) }\KeywordTok{length}\NormalTok{(x}\OperatorTok{$}\NormalTok{producao_bibiografica}\OperatorTok{$}\NormalTok{CAPITULO_DE_LIVRO}\OperatorTok{$}\NormalTok{ano)))}
\end{Highlighting}
\end{Shaded}

\begin{verbatim}
## [1] 16
\end{verbatim}

\begin{Shaded}
\begin{Highlighting}[]
\KeywordTok{sum}\NormalTok{(}\KeywordTok{sapply}\NormalTok{(unb.prof, }\ControlFlowTok{function}\NormalTok{(x) }\KeywordTok{length}\NormalTok{(x}\OperatorTok{$}\NormalTok{producao_bibiografica}\OperatorTok{$}\NormalTok{LIVRO}\OperatorTok{$}\NormalTok{ano)))}
\end{Highlighting}
\end{Shaded}

\begin{verbatim}
## [1] 13
\end{verbatim}

\begin{Shaded}
\begin{Highlighting}[]
\KeywordTok{sum}\NormalTok{(}\KeywordTok{sapply}\NormalTok{(unb.prof, }\ControlFlowTok{function}\NormalTok{(x) }\KeywordTok{length}\NormalTok{(x}\OperatorTok{$}\NormalTok{producao_bibiografica}\OperatorTok{$}\NormalTok{PERIODICO}\OperatorTok{$}\NormalTok{ano)))}
\end{Highlighting}
\end{Shaded}

\begin{verbatim}
## [1] 642
\end{verbatim}

\begin{Shaded}
\begin{Highlighting}[]
\KeywordTok{sum}\NormalTok{(}\KeywordTok{sapply}\NormalTok{(unb.prof, }\ControlFlowTok{function}\NormalTok{(x) }\KeywordTok{length}\NormalTok{(x}\OperatorTok{$}\NormalTok{producao_bibiografica}\OperatorTok{$}\NormalTok{TEXTO_EM_JORNAIS}\OperatorTok{$}\NormalTok{ano)))}
\end{Highlighting}
\end{Shaded}

\begin{verbatim}
## [1] 1
\end{verbatim}

\begin{Shaded}
\begin{Highlighting}[]
\CommentTok{# Número de pessoas por quantitativo de produções por pessoa 0 = 1; 1 = 2...}
\KeywordTok{table}\NormalTok{(}\KeywordTok{unlist}\NormalTok{(}\KeywordTok{sapply}\NormalTok{(unb.prof, }\ControlFlowTok{function}\NormalTok{(x) }\KeywordTok{length}\NormalTok{(x}\OperatorTok{$}\NormalTok{producao_bibiografica}\OperatorTok{$}\NormalTok{ARTIGO_ACEITO}\OperatorTok{$}\NormalTok{ano))))}
\end{Highlighting}
\end{Shaded}

\begin{verbatim}
## 
##  0  1 
## 37  7
\end{verbatim}

\begin{Shaded}
\begin{Highlighting}[]
\KeywordTok{table}\NormalTok{(}\KeywordTok{unlist}\NormalTok{(}\KeywordTok{sapply}\NormalTok{(unb.prof, }\ControlFlowTok{function}\NormalTok{(x) }\KeywordTok{length}\NormalTok{(x}\OperatorTok{$}\NormalTok{producao_bibiografica}\OperatorTok{$}\NormalTok{CAPITULO_DE_LIVRO}\OperatorTok{$}\NormalTok{ano))))}
\end{Highlighting}
\end{Shaded}

\begin{verbatim}
## 
##  0  1  2  3 
## 33  7  3  1
\end{verbatim}

\begin{Shaded}
\begin{Highlighting}[]
\KeywordTok{table}\NormalTok{(}\KeywordTok{unlist}\NormalTok{(}\KeywordTok{sapply}\NormalTok{(unb.prof, }\ControlFlowTok{function}\NormalTok{(x) }\KeywordTok{length}\NormalTok{(x}\OperatorTok{$}\NormalTok{producao_bibiografica}\OperatorTok{$}\NormalTok{LIVRO}\OperatorTok{$}\NormalTok{ano))))}
\end{Highlighting}
\end{Shaded}

\begin{verbatim}
## 
##  0  1  2  7 
## 38  4  1  1
\end{verbatim}

\begin{Shaded}
\begin{Highlighting}[]
\KeywordTok{table}\NormalTok{(}\KeywordTok{unlist}\NormalTok{(}\KeywordTok{sapply}\NormalTok{(unb.prof, }\ControlFlowTok{function}\NormalTok{(x) }\KeywordTok{length}\NormalTok{(x}\OperatorTok{$}\NormalTok{producao_bibiografica}\OperatorTok{$}\NormalTok{PERIODICO}\OperatorTok{$}\NormalTok{ano))))}
\end{Highlighting}
\end{Shaded}

\begin{verbatim}
## 
##  1  2  3  4  5  6  7  8  9 10 11 12 14 15 17 21 22 23 28 36 41 63 78 
##  1  3  2  4  3  2  2  1  3  1  1  5  3  1  1  3  1  1  2  1  1  1  1
\end{verbatim}

\begin{Shaded}
\begin{Highlighting}[]
\KeywordTok{table}\NormalTok{(}\KeywordTok{unlist}\NormalTok{(}\KeywordTok{sapply}\NormalTok{(unb.prof, }\ControlFlowTok{function}\NormalTok{(x) }\KeywordTok{length}\NormalTok{(x}\OperatorTok{$}\NormalTok{producao_bibiografica}\OperatorTok{$}\NormalTok{TEXTO_EM_JORNAIS}\OperatorTok{$}\NormalTok{ano))))}
\end{Highlighting}
\end{Shaded}

\begin{verbatim}
## 
##  0  1 
## 43  1
\end{verbatim}

\begin{Shaded}
\begin{Highlighting}[]
\CommentTok{# Número de produções por ano}
\KeywordTok{table}\NormalTok{(}\KeywordTok{unlist}\NormalTok{(}\KeywordTok{sapply}\NormalTok{(unb.prof, }\ControlFlowTok{function}\NormalTok{(x) (x}\OperatorTok{$}\NormalTok{producao_bibiografica}\OperatorTok{$}\NormalTok{ARTIGO_ACEITO}\OperatorTok{$}\NormalTok{ano))))}
\end{Highlighting}
\end{Shaded}

\begin{verbatim}
## 
## 2013 2016 2017 
##    1    1    5
\end{verbatim}

\begin{Shaded}
\begin{Highlighting}[]
\KeywordTok{table}\NormalTok{(}\KeywordTok{unlist}\NormalTok{(}\KeywordTok{sapply}\NormalTok{(unb.prof, }\ControlFlowTok{function}\NormalTok{(x) (x}\OperatorTok{$}\NormalTok{producao_bibiografica}\OperatorTok{$}\NormalTok{CAPITULO_DE_LIVRO}\OperatorTok{$}\NormalTok{ano))))}
\end{Highlighting}
\end{Shaded}

\begin{verbatim}
## 
## 2010 2011 2012 2013 2014 2015 2016 2017 
##    3    1    1    1    2    4    1    3
\end{verbatim}

\begin{Shaded}
\begin{Highlighting}[]
\KeywordTok{table}\NormalTok{(}\KeywordTok{unlist}\NormalTok{(}\KeywordTok{sapply}\NormalTok{(unb.prof, }\ControlFlowTok{function}\NormalTok{(x) (x}\OperatorTok{$}\NormalTok{producao_bibiografica}\OperatorTok{$}\NormalTok{LIVRO}\OperatorTok{$}\NormalTok{ano))))}
\end{Highlighting}
\end{Shaded}

\begin{verbatim}
## 
## 2010 2011 2012 2013 2014 2015 2017 
##    1    1    1    3    2    1    4
\end{verbatim}

\begin{Shaded}
\begin{Highlighting}[]
\KeywordTok{table}\NormalTok{(}\KeywordTok{unlist}\NormalTok{(}\KeywordTok{sapply}\NormalTok{(unb.prof, }\ControlFlowTok{function}\NormalTok{(x) (x}\OperatorTok{$}\NormalTok{producao_bibiografica}\OperatorTok{$}\NormalTok{PERIODICO}\OperatorTok{$}\NormalTok{ano))))}
\end{Highlighting}
\end{Shaded}

\begin{verbatim}
## 
## 2010 2011 2012 2013 2014 2015 2016 2017 
##   69   70   62   81   87   86   92   95
\end{verbatim}

\begin{Shaded}
\begin{Highlighting}[]
\KeywordTok{table}\NormalTok{(}\KeywordTok{unlist}\NormalTok{(}\KeywordTok{sapply}\NormalTok{(unb.prof, }\ControlFlowTok{function}\NormalTok{(x) (x}\OperatorTok{$}\NormalTok{producao_bibiografica}\OperatorTok{$}\NormalTok{TEXTO_EM_JORNAIS}\OperatorTok{$}\NormalTok{ano))))}
\end{Highlighting}
\end{Shaded}

\begin{verbatim}
## 
## 2014 
##    1
\end{verbatim}

\begin{Shaded}
\begin{Highlighting}[]
\CommentTok{# Número de pessoas que realizaram diferentes tipos de orientações}
\KeywordTok{length}\NormalTok{(}\KeywordTok{unlist}\NormalTok{(}\KeywordTok{sapply}\NormalTok{(unb.prof, }\ControlFlowTok{function}\NormalTok{(x) }\KeywordTok{names}\NormalTok{(x}\OperatorTok{$}\NormalTok{orientacoes_academicas))))}
\end{Highlighting}
\end{Shaded}

\begin{verbatim}
## [1] 130
\end{verbatim}

\begin{Shaded}
\begin{Highlighting}[]
\CommentTok{# Número de pessoas por tipo de orientação}
\KeywordTok{table}\NormalTok{(}\KeywordTok{unlist}\NormalTok{(}\KeywordTok{sapply}\NormalTok{(unb.prof, }\ControlFlowTok{function}\NormalTok{(x) }\KeywordTok{names}\NormalTok{(x}\OperatorTok{$}\NormalTok{orientacoes_academicas))))}
\end{Highlighting}
\end{Shaded}

\begin{verbatim}
## 
##               ORIENTACAO_CONCLUIDA_DOUTORADO 
##                                           23 
##                ORIENTACAO_CONCLUIDA_MESTRADO 
##                                           34 
##           ORIENTACAO_CONCLUIDA_POS_DOUTORADO 
##                                            8 
##            ORIENTACAO_EM_ANDAMENTO_DOUTORADO 
##                                           26 
## ORIENTACAO_EM_ANDAMENTO_INICIACAO_CIENTIFICA 
##                                            9 
##             ORIENTACAO_EM_ANDAMENTO_MESTRADO 
##                                            9 
##                OUTRAS_ORIENTACOES_CONCLUIDAS 
##                                           21
\end{verbatim}

\begin{Shaded}
\begin{Highlighting}[]
\CommentTok{#Número de orientações concluidas}
\KeywordTok{sum}\NormalTok{(}\KeywordTok{sapply}\NormalTok{(unb.prof, }\ControlFlowTok{function}\NormalTok{(x) }\KeywordTok{length}\NormalTok{(x}\OperatorTok{$}\NormalTok{orientacoes_academicas}\OperatorTok{$}\NormalTok{ORIENTACAO_CONCLUIDA_MESTRADO}\OperatorTok{$}\NormalTok{ano)))}
\end{Highlighting}
\end{Shaded}

\begin{verbatim}
## [1] 136
\end{verbatim}

\begin{Shaded}
\begin{Highlighting}[]
\KeywordTok{sum}\NormalTok{(}\KeywordTok{sapply}\NormalTok{(unb.prof, }\ControlFlowTok{function}\NormalTok{(x) }\KeywordTok{length}\NormalTok{(x}\OperatorTok{$}\NormalTok{orientacoes_academicas}\OperatorTok{$}\NormalTok{ORIENTACAO_CONCLUIDA_DOUTORADO}\OperatorTok{$}\NormalTok{ano)))}
\end{Highlighting}
\end{Shaded}

\begin{verbatim}
## [1] 96
\end{verbatim}

\begin{Shaded}
\begin{Highlighting}[]
\KeywordTok{sum}\NormalTok{(}\KeywordTok{sapply}\NormalTok{(unb.prof, }\ControlFlowTok{function}\NormalTok{(x) }\KeywordTok{length}\NormalTok{(x}\OperatorTok{$}\NormalTok{orientacoes_academicas}\OperatorTok{$}\NormalTok{ORIENTACAO_CONCLUIDA_POS_DOUTORADO}\OperatorTok{$}\NormalTok{ano)))}
\end{Highlighting}
\end{Shaded}

\begin{verbatim}
## [1] 18
\end{verbatim}

\begin{Shaded}
\begin{Highlighting}[]
\CommentTok{# Número de pessoas por quantitativo de orientações por pessoa 0 = 1; 1 = 2...}
\KeywordTok{table}\NormalTok{(}\KeywordTok{unlist}\NormalTok{(}\KeywordTok{sapply}\NormalTok{(unb.prof, }\ControlFlowTok{function}\NormalTok{(x) }\KeywordTok{length}\NormalTok{(x}\OperatorTok{$}\NormalTok{orientacoes_academicas}\OperatorTok{$}\NormalTok{ORIENTACAO_CONCLUIDA_MESTRADO}\OperatorTok{$}\NormalTok{ano))))}
\end{Highlighting}
\end{Shaded}

\begin{verbatim}
## 
##  0  1  2  3  4  5  6  7  8  9 11 13 
## 10 11  5  4  3  1  2  2  2  1  2  1
\end{verbatim}

\begin{Shaded}
\begin{Highlighting}[]
\KeywordTok{table}\NormalTok{(}\KeywordTok{unlist}\NormalTok{(}\KeywordTok{sapply}\NormalTok{(unb.prof, }\ControlFlowTok{function}\NormalTok{(x) }\KeywordTok{length}\NormalTok{(x}\OperatorTok{$}\NormalTok{orientacoes_academicas}\OperatorTok{$}\NormalTok{ORIENTACAO_CONCLUIDA_DOUTORADO}\OperatorTok{$}\NormalTok{ano))))}
\end{Highlighting}
\end{Shaded}

\begin{verbatim}
## 
##  0  1  2  3  4  5  6  7 12 
## 21  3  1  7  4  2  3  2  1
\end{verbatim}

\begin{Shaded}
\begin{Highlighting}[]
\KeywordTok{table}\NormalTok{(}\KeywordTok{unlist}\NormalTok{(}\KeywordTok{sapply}\NormalTok{(unb.prof, }\ControlFlowTok{function}\NormalTok{(x) }\KeywordTok{length}\NormalTok{(x}\OperatorTok{$}\NormalTok{orientacoes_academicas}\OperatorTok{$}\NormalTok{ORIENTACAO_CONCLUIDA_POS_DOUTORADO}\OperatorTok{$}\NormalTok{ano))))}
\end{Highlighting}
\end{Shaded}

\begin{verbatim}
## 
##  0  1  2  3  4 
## 36  3  2  1  2
\end{verbatim}

\begin{Shaded}
\begin{Highlighting}[]
\CommentTok{# Número de orientações por ano}
\KeywordTok{table}\NormalTok{(}\KeywordTok{unlist}\NormalTok{(}\KeywordTok{sapply}\NormalTok{(unb.prof, }\ControlFlowTok{function}\NormalTok{(x) (x}\OperatorTok{$}\NormalTok{orientacoes_academicas}\OperatorTok{$}\NormalTok{ORIENTACAO_CONCLUIDA_MESTRADO}\OperatorTok{$}\NormalTok{ano))))}
\end{Highlighting}
\end{Shaded}

\begin{verbatim}
## 
## 2010 2011 2012 2013 2014 2015 2016 2017 
##   14    9   16   21   22   14   23   17
\end{verbatim}

\begin{Shaded}
\begin{Highlighting}[]
\KeywordTok{table}\NormalTok{(}\KeywordTok{unlist}\NormalTok{(}\KeywordTok{sapply}\NormalTok{(unb.prof, }\ControlFlowTok{function}\NormalTok{(x) (x}\OperatorTok{$}\NormalTok{orientacoes_academicas}\OperatorTok{$}\NormalTok{ORIENTACAO_CONCLUIDA_DOUTORADO}\OperatorTok{$}\NormalTok{ano))))}
\end{Highlighting}
\end{Shaded}

\begin{verbatim}
## 
## 2010 2011 2012 2013 2014 2015 2016 2017 
##   18   14    7   14    9    5   16   13
\end{verbatim}

\begin{Shaded}
\begin{Highlighting}[]
\KeywordTok{table}\NormalTok{(}\KeywordTok{unlist}\NormalTok{(}\KeywordTok{sapply}\NormalTok{(unb.prof, }\ControlFlowTok{function}\NormalTok{(x) (x}\OperatorTok{$}\NormalTok{orientacoes_academicas}\OperatorTok{$}\NormalTok{ORIENTACAO_CONCLUIDA_POS_DOUTORADO}\OperatorTok{$}\NormalTok{ano))))}
\end{Highlighting}
\end{Shaded}

\begin{verbatim}
## 
## 2010 2011 2012 2013 2014 2015 2016 2017 
##    1    4    2    2    3    2    3    1
\end{verbatim}

\subsubsection{Arquivo Publicação}\label{arquivo-publicaaao}

\begin{Shaded}
\begin{Highlighting}[]
\CommentTok{# Visualizar a estrutura do arquivo de Publicacao}
\CommentTok{#jsonedit(unb.pub)}
\CommentTok{#Criando um data-frame com todos os anos}
\NormalTok{unb.pub.df <-}\StringTok{ }\KeywordTok{data.frame}\NormalTok{()}
\ControlFlowTok{for}\NormalTok{ (i }\ControlFlowTok{in} \DecValTok{1}\OperatorTok{:}\KeywordTok{length}\NormalTok{(unb.pub[[}\DecValTok{1}\NormalTok{]]))}
\NormalTok{  unb.pub.df <-}\StringTok{ }\KeywordTok{rbind}\NormalTok{(unb.pub.df, unb.pub}\OperatorTok{$}\NormalTok{PERIODICO[[i]])}
\KeywordTok{glimpse}\NormalTok{(unb.pub.df)}
\end{Highlighting}
\end{Shaded}

\begin{verbatim}
## Observations: 569
## Variables: 10
## $ natureza           <chr> "COMPLETO", "COMPLETO", "COMPLETO", "COMPLE...
## $ titulo             <chr> "Symmetric 1-factorizations of the complete...
## $ periodico          <chr> "European Journal of Combinatorics (Print)"...
## $ ano                <chr> "2010", "2010", "2010", "2010", "2010", "20...
## $ volume             <chr> "31", "13", "10", "95", "159", "13", "323",...
## $ issn               <chr> "01956698", "14335883", "15361365", "000388...
## $ paginas            <chr> "1410 - 1418", "649 - 657", "757 - 770", "2...
## $ doi                <chr> "10.1016/j.ejc.2009.12.003", "10.1515/JGT.2...
## $ autores            <list> [<"Pasotti, Anita", "Pellegrini, Marco Ant...
## $ `autores-endogeno` <list> ["0293171472213095", "0293171472213095", "...
\end{verbatim}

\begin{Shaded}
\begin{Highlighting}[]
\CommentTok{# Limpando o data-frame de listas}
\NormalTok{unb.pub.df}\OperatorTok{$}\NormalTok{autores <-}\StringTok{ }\KeywordTok{gsub}\NormalTok{(}\StringTok{"}\CharTok{\textbackslash{}"}\StringTok{,}\CharTok{\textbackslash{}"}\StringTok{|}\CharTok{\textbackslash{}"}\StringTok{, }\CharTok{\textbackslash{}"}\StringTok{"}\NormalTok{, }\StringTok{"; "}\NormalTok{, unb.pub.df}\OperatorTok{$}\NormalTok{autores)}
\NormalTok{unb.pub.df}\OperatorTok{$}\NormalTok{autores <-}\StringTok{ }\KeywordTok{gsub}\NormalTok{(}\StringTok{"}\CharTok{\textbackslash{}"}\StringTok{|c}\CharTok{\textbackslash{}\textbackslash{}}\StringTok{(|}\CharTok{\textbackslash{}\textbackslash{}}\StringTok{)"}\NormalTok{, }\StringTok{""}\NormalTok{, unb.pub.df}\OperatorTok{$}\NormalTok{autores)}
\NormalTok{unb.pub.df}\OperatorTok{$}\StringTok{`}\DataTypeTok{autores-endogeno}\StringTok{`}\NormalTok{ <-}\StringTok{ }\KeywordTok{gsub}\NormalTok{(}\StringTok{","}\NormalTok{, }\StringTok{";"}\NormalTok{, unb.pub.df}\OperatorTok{$}\StringTok{`}\DataTypeTok{autores-endogeno}\StringTok{`}\NormalTok{)}
\NormalTok{unb.pub.df}\OperatorTok{$}\StringTok{`}\DataTypeTok{autores-endogeno}\StringTok{`}\NormalTok{ <-}\StringTok{ }\KeywordTok{gsub}\NormalTok{(}\StringTok{"}\CharTok{\textbackslash{}"}\StringTok{|c}\CharTok{\textbackslash{}\textbackslash{}}\StringTok{(|}\CharTok{\textbackslash{}\textbackslash{}}\StringTok{)"}\NormalTok{, }\StringTok{""}\NormalTok{, unb.pub.df}\OperatorTok{$}\StringTok{`}\DataTypeTok{autores-endogeno}\StringTok{`}\NormalTok{)}
\KeywordTok{glimpse}\NormalTok{(unb.pub.df)}
\end{Highlighting}
\end{Shaded}

\begin{verbatim}
## Observations: 569
## Variables: 10
## $ natureza           <chr> "COMPLETO", "COMPLETO", "COMPLETO", "COMPLE...
## $ titulo             <chr> "Symmetric 1-factorizations of the complete...
## $ periodico          <chr> "European Journal of Combinatorics (Print)"...
## $ ano                <chr> "2010", "2010", "2010", "2010", "2010", "20...
## $ volume             <chr> "31", "13", "10", "95", "159", "13", "323",...
## $ issn               <chr> "01956698", "14335883", "15361365", "000388...
## $ paginas            <chr> "1410 - 1418", "649 - 657", "757 - 770", "2...
## $ doi                <chr> "10.1016/j.ejc.2009.12.003", "10.1515/JGT.2...
## $ autores            <chr> "Pasotti, Anita; Pellegrini, Marco Antonio"...
## $ `autores-endogeno` <chr> "0293171472213095", "0293171472213095", "04...
\end{verbatim}

\subsubsection{Arquivo Orientação}\label{arquivo-orientaaao}

\begin{Shaded}
\begin{Highlighting}[]
\CommentTok{#Orientação}
\CommentTok{#Visualizar a estrutura do json no painel Viewer}
\CommentTok{#jsonedit(unb.adv)}
\CommentTok{#Reunir todos os anos e orientações concluidas em um mesmo data-frame}
\NormalTok{unb.adv.tipo.df <-}\StringTok{ }\KeywordTok{data.frame}\NormalTok{(); unb.adv.df <-}\StringTok{ }\KeywordTok{data.frame}\NormalTok{()}
\ControlFlowTok{for}\NormalTok{ (i }\ControlFlowTok{in} \DecValTok{1}\OperatorTok{:}\KeywordTok{length}\NormalTok{(unb.adv[[}\DecValTok{1}\NormalTok{]]))}
\NormalTok{  unb.adv.tipo.df <-}\StringTok{ }\KeywordTok{rbind}\NormalTok{(unb.adv.tipo.df, unb.adv}\OperatorTok{$}\NormalTok{ORIENTACAO_CONCLUIDA_POS_DOUTORADO[[i]])}
\NormalTok{unb.adv.df <-}\StringTok{ }\KeywordTok{rbind}\NormalTok{(unb.adv.df, unb.adv.tipo.df); unb.adv.tipo.df <-}\StringTok{ }\KeywordTok{data.frame}\NormalTok{()}
\ControlFlowTok{for}\NormalTok{ (i }\ControlFlowTok{in} \DecValTok{1}\OperatorTok{:}\KeywordTok{length}\NormalTok{(unb.adv[[}\DecValTok{1}\NormalTok{]]))}
\NormalTok{  unb.adv.tipo.df <-}\StringTok{ }\KeywordTok{rbind}\NormalTok{(unb.adv.tipo.df, unb.adv}\OperatorTok{$}\NormalTok{ORIENTACAO_CONCLUIDA_DOUTORADO[[i]])}
\NormalTok{unb.adv.df <-}\StringTok{ }\KeywordTok{rbind}\NormalTok{(unb.adv.df, unb.adv.tipo.df); unb.adv.tipo.df <-}\StringTok{ }\KeywordTok{data.frame}\NormalTok{()}
\ControlFlowTok{for}\NormalTok{ (i }\ControlFlowTok{in} \DecValTok{1}\OperatorTok{:}\KeywordTok{length}\NormalTok{(unb.adv[[}\DecValTok{1}\NormalTok{]]))}
\NormalTok{  unb.adv.tipo.df <-}\StringTok{ }\KeywordTok{rbind}\NormalTok{(unb.adv.tipo.df, unb.adv}\OperatorTok{$}\NormalTok{ORIENTACAO_CONCLUIDA_MESTRADO[[i]])}
\NormalTok{unb.adv.df <-}\StringTok{ }\KeywordTok{rbind}\NormalTok{(unb.adv.df, unb.adv.tipo.df)}
\KeywordTok{glimpse}\NormalTok{(unb.adv.df)}
\end{Highlighting}
\end{Shaded}

\begin{verbatim}
## Observations: 245
## Variables: 13
## $ natureza                    <chr> "Supervisão de pós-doutorado", "Su...
## $ titulo                      <chr> "", "", "", "", "", "", "", "", ""...
## $ ano                         <chr> "2010", "2011", "2011", "2011", "2...
## $ id_lattes_aluno             <chr> "", "", "", "", "", "", "", "", ""...
## $ nome_aluno                  <chr> "Ilir Snopche", "ACCIARRI CRISTINA...
## $ instituicao                 <chr> "Universidade de Brasília", "Unive...
## $ curso                       <chr> "", "", "", "", "", "", "", "", ""...
## $ codigo_do_curso             <chr> "", "", "", "", "", "", "", "", ""...
## $ bolsa                       <chr> "SIM", "SIM", "SIM", "SIM", "SIM",...
## $ agencia_financiadora        <chr> "Conselho Nacional de Desenvolvime...
## $ codigo_agencia_financiadora <chr> "002200000000", "002200000000", "0...
## $ nome_orientadores           <list> ["Pavel Zalesski", "Pavel Shumyat...
## $ id_lattes_orientadores      <list> ["9177572441291520", "15954372903...
\end{verbatim}

\begin{Shaded}
\begin{Highlighting}[]
\CommentTok{#Transformar as colunas de listas em caracteres eliminando c("")}
\NormalTok{unb.adv.df}\OperatorTok{$}\NormalTok{nome_orientadores <-}\StringTok{ }\KeywordTok{gsub}\NormalTok{(}\StringTok{"}\CharTok{\textbackslash{}"}\StringTok{|c}\CharTok{\textbackslash{}\textbackslash{}}\StringTok{(|}\CharTok{\textbackslash{}\textbackslash{}}\StringTok{)"}\NormalTok{, }\StringTok{""}\NormalTok{, unb.adv.df}\OperatorTok{$}\NormalTok{nome_orientadores)}
\NormalTok{unb.adv.df}\OperatorTok{$}\NormalTok{id_lattes_orientadores <-}\StringTok{ }\KeywordTok{gsub}\NormalTok{(}\StringTok{"}\CharTok{\textbackslash{}"}\StringTok{|c}\CharTok{\textbackslash{}\textbackslash{}}\StringTok{(|}\CharTok{\textbackslash{}\textbackslash{}}\StringTok{)"}\NormalTok{, }\StringTok{""}\NormalTok{, unb.adv.df}\OperatorTok{$}\NormalTok{id_lattes_orientadores)}
\CommentTok{#Separar as colunas com dois orientadores}
\NormalTok{unb.adv.df <-}\StringTok{ }\KeywordTok{separate}\NormalTok{(unb.adv.df, nome_orientadores, }\DataTypeTok{into =} \KeywordTok{c}\NormalTok{(}\StringTok{"ori1"}\NormalTok{, }\StringTok{"ori2"}\NormalTok{), }\DataTypeTok{sep =} \StringTok{","}\NormalTok{)}
\end{Highlighting}
\end{Shaded}

\begin{verbatim}
## Warning: Expected 2 pieces. Missing pieces filled with `NA` in 240 rows [1,
## 2, 3, 4, 5, 6, 7, 8, 9, 10, 11, 12, 13, 14, 15, 17, 18, 19, 20, 21, ...].
\end{verbatim}

\begin{Shaded}
\begin{Highlighting}[]
\NormalTok{unb.adv.df <-}\StringTok{ }\KeywordTok{separate}\NormalTok{(unb.adv.df, id_lattes_orientadores, }\DataTypeTok{into =} \KeywordTok{c}\NormalTok{(}\StringTok{"idLattes1"}\NormalTok{, }\StringTok{"idLattes2"}\NormalTok{), }\DataTypeTok{sep =} \StringTok{","}\NormalTok{)}
\end{Highlighting}
\end{Shaded}

\begin{verbatim}
## Warning: Expected 2 pieces. Missing pieces filled with `NA` in 240 rows [1,
## 2, 3, 4, 5, 6, 7, 8, 9, 10, 11, 12, 13, 14, 15, 17, 18, 19, 20, 21, ...].
\end{verbatim}

\begin{Shaded}
\begin{Highlighting}[]
\CommentTok{#Numero de orientacoes por ano}
\KeywordTok{table}\NormalTok{(unb.adv.df}\OperatorTok{$}\NormalTok{ano)}
\end{Highlighting}
\end{Shaded}

\begin{verbatim}
## 
## 2010 2011 2012 2013 2014 2015 2016 2017 
##   30   26   25   37   34   21   41   31
\end{verbatim}

\begin{Shaded}
\begin{Highlighting}[]
\CommentTok{#Tabela com nome de professor e numero de orientacoes}
\KeywordTok{head}\NormalTok{(}\KeywordTok{sort}\NormalTok{(}\KeywordTok{table}\NormalTok{(}\KeywordTok{rbind}\NormalTok{(unb.adv.df}\OperatorTok{$}\NormalTok{ori1, unb.adv.df}\OperatorTok{$}\NormalTok{ori2)), }\DataTypeTok{decreasing =} \OtherTok{TRUE}\NormalTok{), }\DecValTok{20}\NormalTok{)}
\end{Highlighting}
\end{Shaded}

\begin{verbatim}
## 
##                 Mauricio Ayala Rincon 
##                                    20 
##   Giovany de Jesus Malcher Figueiredo 
##                                    17 
##                        Keti Tenenblat 
##                                    16 
##                Diego Marques Ferreira 
##                                    14 
##              Marco Antonio Pellegrini 
##                                    11 
##                       Ricardo Ruviaro 
##                                    11 
##              Leandro Martins Cioletti 
##                                     9 
##             Ricardo Parreira da Silva 
##                                     9 
##                  Chang Chung Yu Dorea 
##                                     8 
##                Hemar Teixeira Godinho 
##                                     8 
##               Liliane de Almeida Maia 
##                                     8 
##                      Pavel Shumyatsky 
##                                     8 
##                   Alexei Krassilnikov 
##                                     7 
##     Carlos Alberto Pereira dos Santos 
##                                     7 
##                 João Paulo dos Santos 
##                                     7 
## Luciana Maria Dias de Ávila Rodrigues 
##                                     7 
##            Aline Gomes da Silva Pinto 
##                                     6 
##             Marcelo Fernandes Furtado 
##                                     6 
##                     Norai Romeu Rocco 
##                                     6 
##                        Pavel Zalesski 
##                                     6
\end{verbatim}

\subsection{CRISP-DM Fase.Atividade 2.4 - Verificação da qualidade dos
dados.}\label{crisp-dm-fase.atividade-2.4---verificaaao-da-qualidade-dos-dados.}

Como já informado, a verificação da qualidade dos dados envolve
responder se os dados disponíveis estão realmente completos.

As informações disponíveis são suficientes para o trabalho proposto?

Neste projeto, a verificação da qualidade dos dados é
responsabilidade dos alunos.

\section{\texorpdfstring{CRISP-DM Fase 3 - \textbf{Preparação dos
Dados}}{CRISP-DM Fase 3 - Preparação dos Dados}}\label{crisp-dm-fase-3---preparaaao-dos-dados}

Como já informado, na fase de \textbf{Preparação dos Dados} os
\emph{datasets} que serão utilizados em todo o trabalho são
construídos a partir dos dados brutos. Aqui os dados são
“filtrados” retirando-se partes que não interessam e
selecionando-se os “campos” necessários para o trabalho de
mineração.

São 5 as atividades genéricas nesta fase de preparação dos dados, a
seguir divididas em subseções

\subsection{CRISP-DM Fase.Atividade 3.1 - Seleção dos
dados.}\label{crisp-dm-fase.atividade-3.1---seleaao-dos-dados.}

Como já informado, a seleção dos dados envolve identificar quais
dados, da nossa ``montanha de dados'', serão realmente utilizados.

Quais variáveis dos dados brutos serão convertidas para o
\emph{dataset}?

Não é raro cometer o erro de selecionar dados para um modelo preditivo
com base em uma falsa ideia de que aqueles dados contém a resposta para
o modelo que se quer construir. Surge o cuidado de se separar o sinal do
ruído (Silver, Nate. The Signal and the Noise: Why so many predictions
fail — but some don’t. USA: The Penguin Press HC, 2012.).

\subsection{CRISP-DM Fase.Atividade 3.2 - Limpeza dos
dados}\label{crisp-dm-fase.atividade-3.2---limpeza-dos-dados}

\subsection{CRISP-DM Fase.Atividade 3.3 - Construção dos
dados}\label{crisp-dm-fase.atividade-3.3---construaao-dos-dados}

Como já informado, a construção dos dados envolve a criação de
novas variáveis a partir de outras presentes nos \emph{datasets}.

\begin{Shaded}
\begin{Highlighting}[]
\CommentTok{# Funcoes }

\CommentTok{# converte as colunas de um dataframe tipo lista em tipo character}
\NormalTok{cv_tplista2tpchar <-}\StringTok{ }\ControlFlowTok{function}\NormalTok{( df  ) \{ }
  \ControlFlowTok{for}\NormalTok{( variavel }\ControlFlowTok{in} \KeywordTok{names}\NormalTok{(df)) \{}
    \ControlFlowTok{if}\NormalTok{ (}\KeywordTok{class}\NormalTok{(df[[variavel]]) }\OperatorTok{==}\StringTok{ "list"}\NormalTok{ ) \{}
\NormalTok{      df[[variavel]] <-}\StringTok{ }\KeywordTok{lapply}\NormalTok{(df[[variavel]] ,   }\ControlFlowTok{function}\NormalTok{(x)   }\KeywordTok{lista2texto}\NormalTok{( x  ) ) }
\NormalTok{      df[[variavel]] <-}\StringTok{ }\KeywordTok{as.character}\NormalTok{( df[[variavel]] )}
\NormalTok{    \}}
\NormalTok{  \}}
  \KeywordTok{return}\NormalTok{(df)}
\NormalTok{\}}
\NormalTok{###}


\CommentTok{# converte o conteudo de lista em array de characters}
\NormalTok{lista2texto <-}\StringTok{ }\ControlFlowTok{function}\NormalTok{( lista  ) \{}
  \ControlFlowTok{if}\NormalTok{(}\KeywordTok{is.null}\NormalTok{(lista)) \{}
    \KeywordTok{return}\NormalTok{ ( }\OtherTok{NULL}\NormalTok{ )}
\NormalTok{  \}}
\NormalTok{  saida <-}\StringTok{ ""}
  \ControlFlowTok{for}\NormalTok{( j }\ControlFlowTok{in} \DecValTok{1}\OperatorTok{:}\KeywordTok{length}\NormalTok{(lista)) \{ }
    \ControlFlowTok{for}\NormalTok{( i }\ControlFlowTok{in} \DecValTok{1}\OperatorTok{:}\KeywordTok{length}\NormalTok{(lista[[j]]) ) \{}
\NormalTok{      elemento <-}\StringTok{ }\NormalTok{lista[[j]][i] }
      \ControlFlowTok{if}\NormalTok{( }\OperatorTok{!}\KeywordTok{is.null}\NormalTok{(elemento)) \{ }
        \ControlFlowTok{if}\NormalTok{( i }\OperatorTok{==}\StringTok{ }\KeywordTok{length}\NormalTok{(lista[[j]]) }\OperatorTok{&}\StringTok{ }\NormalTok{j }\OperatorTok{==}\StringTok{ }\KeywordTok{length}\NormalTok{(lista)  ) \{ }
          \CommentTok{# se for o ultimo elemento nao coloque o ponto e virgula no final            }
\NormalTok{          saida <-}\StringTok{ }\KeywordTok{paste0}\NormalTok{( saida , elemento  )}
\NormalTok{        \} }\ControlFlowTok{else}\NormalTok{ \{}
          \CommentTok{# enquanto nao for o ultimo coloque ; separando os elementos concatenados }
\NormalTok{          saida <-}\StringTok{ }\KeywordTok{paste0}\NormalTok{( saida , elemento , }\DataTypeTok{sep =} \StringTok{" ; "}\NormalTok{)}
\NormalTok{        \}}
\NormalTok{      \}  }
\NormalTok{    \}}
\NormalTok{  \}}
  \KeywordTok{return}\NormalTok{( saida )}
\NormalTok{\}}

\CommentTok{# Converte producao elattes separada por anos em um unico dataframe }
\NormalTok{converte_producao2dataframe<-}\StringTok{ }\ControlFlowTok{function}\NormalTok{( lista_producao ) \{}
\NormalTok{  df_saida <-}\StringTok{ }\OtherTok{NULL} 
  
  \ControlFlowTok{for}\NormalTok{( ano }\ControlFlowTok{in} \KeywordTok{names}\NormalTok{(lista_producao)) \{}
\NormalTok{    df_saida <-}\StringTok{ }\KeywordTok{rbind}\NormalTok{(df_saida , lista_producao[[ano]])}
\NormalTok{  \}}
  
  \CommentTok{# converte tipo lista em array de character }
\NormalTok{  df_saida <-}\StringTok{ }\KeywordTok{cv_tplista2tpchar}\NormalTok{(df_saida)}
  \KeywordTok{return}\NormalTok{(df_saida)}
  

\NormalTok{\}}

\CommentTok{#concatena dois dataframes com  colunas diferentes }
\NormalTok{concatenadf <-}\StringTok{ }\ControlFlowTok{function}\NormalTok{( df1, df2) \{ }
  \CommentTok{#cria colunas de df1 que faltam em df2}
  \ControlFlowTok{for}\NormalTok{( coluna }\ControlFlowTok{in} \KeywordTok{names}\NormalTok{(df1 ) ) \{}
    \ControlFlowTok{if}\NormalTok{( }\OperatorTok{!}\KeywordTok{is.element}\NormalTok{(coluna, }\KeywordTok{names}\NormalTok{(df2) )) \{}
\NormalTok{      df2[coluna] <-}\StringTok{ }\OtherTok{NA}
\NormalTok{    \}}
\NormalTok{  \}}
  
  \CommentTok{#cria colunas de df2 que faltam em df1  }
  \ControlFlowTok{for}\NormalTok{( coluna }\ControlFlowTok{in} \KeywordTok{names}\NormalTok{(df2 ) ) \{}
    
    \ControlFlowTok{if}\NormalTok{( }\OperatorTok{!}\KeywordTok{is.element}\NormalTok{(coluna, }\KeywordTok{names}\NormalTok{(df1) )) \{}
\NormalTok{      df1[coluna] <-}\StringTok{ }\OtherTok{NA}
\NormalTok{    \}}
\NormalTok{  \}}
  
  
  \CommentTok{#faz o rbind dos dois dataframes }
\NormalTok{  df_final <-}\StringTok{ }\KeywordTok{rbind}\NormalTok{(df1 , df2)}
  \KeywordTok{return}\NormalTok{(df_final)}
  
\NormalTok{\}}

\CommentTok{# Extracao dos perfis dos professores }

\NormalTok{extrai_1perfil <-}\StringTok{ }\ControlFlowTok{function}\NormalTok{( professor ) \{}
\NormalTok{  idLattes <-}\StringTok{ }\KeywordTok{names}\NormalTok{(professor)}
\NormalTok{  nome <-}\StringTok{ }\NormalTok{professor[[idLattes]]}\OperatorTok{$}\NormalTok{nome   }
\NormalTok{  resumo_cv <-}\StringTok{ }\NormalTok{professor[[idLattes]]}\OperatorTok{$}\NormalTok{resumo_cv }
\NormalTok{  endereco_profissional <-}\StringTok{ }\NormalTok{professor[[idLattes]]}\OperatorTok{$}\NormalTok{endereco_profissional }\CommentTok{#list }
\NormalTok{  instituicao <-}\StringTok{ }\NormalTok{endereco_profissional}\OperatorTok{$}\NormalTok{instituicao}
\NormalTok{  orgao <-}\StringTok{ }\NormalTok{endereco_profissional}\OperatorTok{$}\NormalTok{orgao}
\NormalTok{  unidade <-}\StringTok{ }\NormalTok{endereco_profissional}\OperatorTok{$}\NormalTok{unidade}
\NormalTok{  DDD <-}\StringTok{ }\NormalTok{endereco_profissional}\OperatorTok{$}\NormalTok{DDD}
\NormalTok{  telefone <-}\StringTok{ }\NormalTok{endereco_profissional}\OperatorTok{$}\NormalTok{telefone}
\NormalTok{  bairro <-}\StringTok{ }\NormalTok{endereco_profissional}\OperatorTok{$}\NormalTok{bairro}
\NormalTok{  cep <-}\StringTok{ }\NormalTok{endereco_profissional}\OperatorTok{$}\NormalTok{cep}
\NormalTok{  cidade <-}\StringTok{ }\NormalTok{endereco_profissional}\OperatorTok{$}\NormalTok{cidade}
\NormalTok{  senioridade <-}\StringTok{ }\NormalTok{professor[[idLattes]]}\OperatorTok{$}\NormalTok{senioridade  }
\NormalTok{  df_1perfil <-}\StringTok{ }\KeywordTok{data.frame}\NormalTok{( idLattes , nome, resumo_cv ,instituicao , }
\NormalTok{                           orgao, unidade , DDD, telefone, bairro,cep,cidade , senioridade,}
                           \DataTypeTok{stringsAsFactors =} \OtherTok{FALSE}\NormalTok{)}
  
  \KeywordTok{return}\NormalTok{(df_1perfil)  }
\NormalTok{\}}

\NormalTok{extrai_perfis <-}\StringTok{ }\ControlFlowTok{function}\NormalTok{(jsonProfessores) \{}
\NormalTok{  df_saida <-}\StringTok{ }\KeywordTok{data.frame}\NormalTok{()}
  \ControlFlowTok{for}\NormalTok{( i }\ControlFlowTok{in} \DecValTok{1}\OperatorTok{:}\KeywordTok{length}\NormalTok{(jsonProfessores)) \{}
\NormalTok{    jsonProfessor <-}\StringTok{ }\NormalTok{jsonProfessores[i]}
\NormalTok{    df_professor <-}\StringTok{ }\KeywordTok{extrai_1perfil}\NormalTok{(jsonProfessor)}
    \ControlFlowTok{if}\NormalTok{( }\KeywordTok{nrow}\NormalTok{(df_saida) }\OperatorTok{>}\StringTok{ }\DecValTok{0}\NormalTok{ ) \{}
\NormalTok{      df_saida <-}\StringTok{ }\KeywordTok{rbind}\NormalTok{(df_saida , df_professor)}
\NormalTok{    \} }\ControlFlowTok{else}\NormalTok{ \{}
\NormalTok{      df_saida <-}\StringTok{ }\NormalTok{df_professor }
\NormalTok{    \}}
\NormalTok{  \}}
   
  \KeywordTok{return}\NormalTok{(df_saida)}
\NormalTok{\}}

\CommentTok{# Extracao da producao bibliografica dos professores }

\NormalTok{extrai_1producao <-}\StringTok{ }\ControlFlowTok{function}\NormalTok{(professor) \{}
\NormalTok{  idLattes <-}\StringTok{ }\KeywordTok{names}\NormalTok{(professor)}
\NormalTok{  df_1producao <<-}\StringTok{ }\OtherTok{NULL} 
\NormalTok{  producao_bibliografica <-}\StringTok{ }\NormalTok{professor[[idLattes]]}\OperatorTok{$}\NormalTok{producao_bibiografica  }\CommentTok{#list}
  \ControlFlowTok{for}\NormalTok{( tipo_producao }\ControlFlowTok{in} \KeywordTok{names}\NormalTok{(producao_bibliografica)) \{ }
\NormalTok{    df_temporario <-}\StringTok{ }\KeywordTok{cv_tplista2tpchar}\NormalTok{ ( producao_bibliografica[[tipo_producao]]) }
\NormalTok{    df_temporario}\OperatorTok{$}\NormalTok{tipo_producao <-}\StringTok{  }\NormalTok{tipo_producao }
\NormalTok{    df_temporario}\OperatorTok{$}\NormalTok{idLattes <-}\StringTok{  }\NormalTok{idLattes}
\NormalTok{    df_1producao <-}\StringTok{ }\KeywordTok{concatenadf}\NormalTok{( df_1producao , df_temporario  )}
\NormalTok{  \}  }
  \KeywordTok{return}\NormalTok{(df_1producao)}
\NormalTok{\}}

\NormalTok{extrai_producoes <-}\StringTok{ }\ControlFlowTok{function}\NormalTok{( jsonProfessores) \{}
\NormalTok{  df_saida <-}\StringTok{ }\KeywordTok{data.frame}\NormalTok{()}
  \ControlFlowTok{for}\NormalTok{( i }\ControlFlowTok{in} \DecValTok{1}\OperatorTok{:}\KeywordTok{length}\NormalTok{(jsonProfessores)) \{}
\NormalTok{    jsonProfessor <-}\StringTok{ }\NormalTok{jsonProfessores[i]}
\NormalTok{    df_producao <-}\StringTok{ }\KeywordTok{extrai_1producao}\NormalTok{(jsonProfessor)}
    \ControlFlowTok{if}\NormalTok{( }\KeywordTok{nrow}\NormalTok{(df_saida) }\OperatorTok{>}\StringTok{ }\DecValTok{0}\NormalTok{ ) \{}
\NormalTok{      df_saida <-}\StringTok{ }\KeywordTok{concatenadf}\NormalTok{(df_saida , df_producao)}
\NormalTok{    \} }\ControlFlowTok{else}\NormalTok{ \{}
\NormalTok{      df_saida <-}\StringTok{ }\NormalTok{df_producao }
\NormalTok{    \}}
\NormalTok{  \}}
\NormalTok{  df_saida <-}\StringTok{ }\NormalTok{df_saida }\OperatorTok\StringTok{ }\KeywordTok{filter}\NormalTok{( }\OperatorTok{!}\KeywordTok{is.na}\NormalTok{(tipo_producao))}
  \KeywordTok{return}\NormalTok{(df_saida)  }
\NormalTok{\}}

\CommentTok{# Extracao das orientacoes dos professores }

\NormalTok{extrai_1orientacao <-}\StringTok{ }\ControlFlowTok{function}\NormalTok{(professor) \{}
\NormalTok{  idLattes <-}\StringTok{ }\KeywordTok{names}\NormalTok{(professor)}
\NormalTok{  df_1orientacao <-}\StringTok{ }\OtherTok{NULL}
\NormalTok{  orientacoes_academicas  <-}\StringTok{ }\NormalTok{professor[[idLattes]]}\OperatorTok{$}\NormalTok{orientacoes_academicas  }\CommentTok{#list}
  \ControlFlowTok{for}\NormalTok{( orientacao }\ControlFlowTok{in} \KeywordTok{names}\NormalTok{(orientacoes_academicas )) \{ }
\NormalTok{    df_temporario <-}\StringTok{ }\KeywordTok{cv_tplista2tpchar}\NormalTok{ ( orientacoes_academicas[[orientacao]])}
\NormalTok{    df_temporario}\OperatorTok{$}\NormalTok{orientacao <-}\StringTok{  }\NormalTok{orientacao }
\NormalTok{    df_temporario}\OperatorTok{$}\NormalTok{idLattes <-}\StringTok{  }\NormalTok{idLattes}
\NormalTok{    df_1orientacao <-}\StringTok{ }\KeywordTok{concatenadf}\NormalTok{( df_1orientacao , df_temporario  )}
\NormalTok{  \}  }
  \KeywordTok{return}\NormalTok{(df_1orientacao) }
\NormalTok{\}}

\NormalTok{extrai_orientacoes <-}\StringTok{ }\ControlFlowTok{function}\NormalTok{(jsonProfessores) \{}
\NormalTok{  df_saida <-}\StringTok{ }\KeywordTok{data.frame}\NormalTok{()}
  \ControlFlowTok{for}\NormalTok{( i }\ControlFlowTok{in} \DecValTok{1}\OperatorTok{:}\KeywordTok{length}\NormalTok{(jsonProfessores)) \{}
\NormalTok{    jsonProfessor <-}\StringTok{ }\NormalTok{jsonProfessores[i]}
\NormalTok{    df_orientacao <-}\StringTok{ }\KeywordTok{extrai_1orientacao}\NormalTok{(jsonProfessor)}
    \ControlFlowTok{if}\NormalTok{( }\KeywordTok{nrow}\NormalTok{(df_saida) }\OperatorTok{>}\StringTok{ }\DecValTok{0}\NormalTok{ ) \{}
\NormalTok{      df_saida <-}\StringTok{ }\KeywordTok{concatenadf}\NormalTok{(df_saida , df_orientacao)}
\NormalTok{    \} }\ControlFlowTok{else}\NormalTok{ \{}
\NormalTok{      df_saida <-}\StringTok{ }\NormalTok{df_orientacao}
\NormalTok{    \}}
\NormalTok{  \}}
\NormalTok{  df_saida <-}\StringTok{ }\NormalTok{df_saida }\OperatorTok\StringTok{ }\KeywordTok{filter}\NormalTok{(}\OperatorTok{!}\KeywordTok{is.na}\NormalTok{(idLattes))}
  \KeywordTok{return}\NormalTok{(df_saida)  }
\NormalTok{\}}

\CommentTok{# Extracao das areas de atuacao dos professores }

\NormalTok{extrai_1area_de_atuacao <-}\StringTok{ }\ControlFlowTok{function}\NormalTok{(professor)\{}
\NormalTok{  idLattes <-}\StringTok{ }\KeywordTok{names}\NormalTok{(professor)}
\NormalTok{  df_1area <-}\StringTok{  }\NormalTok{professor[[idLattes]]}\OperatorTok{$}\NormalTok{areas_de_atuacao}
\NormalTok{  df_1area}\OperatorTok{$}\NormalTok{idLattes <-}\StringTok{ }\NormalTok{idLattes}
  \KeywordTok{return}\NormalTok{(df_1area)}
\NormalTok{\}}

\NormalTok{extrai_areas_atuacao <-}\StringTok{ }\ControlFlowTok{function}\NormalTok{(jsonProfessores)\{}
\NormalTok{  df_saida <-}\StringTok{ }\KeywordTok{data.frame}\NormalTok{()}
  \ControlFlowTok{for}\NormalTok{( i }\ControlFlowTok{in} \DecValTok{1}\OperatorTok{:}\KeywordTok{length}\NormalTok{(jsonProfessores)) \{}
\NormalTok{    jsonProfessor <-}\StringTok{ }\NormalTok{jsonProfessores[i]}
\NormalTok{    df_area_atuacao <-}\StringTok{ }\KeywordTok{extrai_1area_de_atuacao}\NormalTok{(jsonProfessor)}
    \ControlFlowTok{if}\NormalTok{( }\KeywordTok{nrow}\NormalTok{(df_saida) }\OperatorTok{>}\StringTok{ }\DecValTok{0}\NormalTok{ ) \{}
\NormalTok{      df_saida <-}\StringTok{ }\KeywordTok{concatenadf}\NormalTok{(df_saida , df_area_atuacao)}
\NormalTok{    \} }\ControlFlowTok{else}\NormalTok{ \{}
\NormalTok{      df_saida <-}\StringTok{ }\NormalTok{df_area_atuacao}
\NormalTok{    \}}
\NormalTok{  \}}
\NormalTok{  df_saida <-}\StringTok{ }\NormalTok{df_saida }\OperatorTok\StringTok{ }\KeywordTok{filter}\NormalTok{( }\OperatorTok{!}\KeywordTok{is.na}\NormalTok{(idLattes))}
  \KeywordTok{return}\NormalTok{(df_saida)   }
\NormalTok{\}}
\NormalTok{########################### Inicio }

\CommentTok{# colocar o diretorio onde está o arquivo json de perfis a serem lidos }
\NormalTok{unb.prof.json <-}\StringTok{ }\KeywordTok{read_file}\NormalTok{(}\StringTok{"Matematica.profile.json"}\NormalTok{)}
\CommentTok{# unb.prof.df.capes <- read.csv("data/PesqPosCapes.csv", }
\CommentTok{#                               sep = ";", header = TRUE, colClasses = "character")}
\NormalTok{unb.prof <-}\StringTok{ }\KeywordTok{fromJSON}\NormalTok{(unb.prof.json)}
\KeywordTok{length}\NormalTok{(unb.prof)}
\end{Highlighting}
\end{Shaded}

\begin{verbatim}
## [1] 44
\end{verbatim}

\begin{Shaded}
\begin{Highlighting}[]
\CommentTok{# extrai perfis dos professores }
\NormalTok{unb.prof.df.professores <-}\StringTok{ }\KeywordTok{extrai_perfis}\NormalTok{(unb.prof)}

\CommentTok{# extrai producao bibliografica de todos os professores }
\NormalTok{unb.prof.df.publicacoes <-}\StringTok{ }\KeywordTok{extrai_producoes}\NormalTok{(unb.prof)}
\end{Highlighting}
\end{Shaded}

\begin{verbatim}
## Warning: package 'bindrcpp' was built under R version 3.4.4
\end{verbatim}

\begin{Shaded}
\begin{Highlighting}[]
\CommentTok{#extrai orientacoes }
\NormalTok{unb.prof.df.orientacoes <-}\StringTok{ }\KeywordTok{extrai_orientacoes}\NormalTok{(unb.prof)}

\CommentTok{#extrai areas de atuacao }
\NormalTok{unb.prof.df.areas.de.atuacao <-}\StringTok{ }\KeywordTok{extrai_areas_atuacao}\NormalTok{(unb.prof)}

\CommentTok{#salva os daframes }
\CommentTok{# save(unb.prof.df.professores, unb.prof.df.publicacoes,}
\CommentTok{#      unb.prof.df.orientacoes, unb.prof.df.areas.de.atuacao, file = "dataframes.Rda")}

\CommentTok{#cria arquivo para análise}
\CommentTok{# unb.prof.df <- data.frame()}
\CommentTok{# unb.prof.df <- unb.prof.df.professores %>% }
\CommentTok{#   select(idLattes, nome, resumo_cv, senioridade) %>% }
\CommentTok{#   left_join(}
\CommentTok{#     unb.prof.df.orientacoes %>% }
\CommentTok{#       select(orientacao, idLattes) %>% }
\CommentTok{#       filter(!grepl("EM_ANDAMENTO", orientacao)) %>% }
\CommentTok{#       group_by(idLattes) %>% }
\CommentTok{#       count(orientacao) %>% }
\CommentTok{#       spread(key = orientacao, value = n), }
\CommentTok{#     by = "idLattes") %>% }
\CommentTok{#   left_join(}
\CommentTok{#     unb.prof.df.publicacoes %>% }
\CommentTok{#       select(tipo_producao, idLattes) %>% }
\CommentTok{#       filter(!grepl("ARTIGO_ACEITO", tipo_producao)) %>% }
\CommentTok{#       group_by(idLattes) %>% }
\CommentTok{#       count(tipo_producao) %>% }
\CommentTok{#       spread(key = tipo_producao, value = n), }
\CommentTok{#     by = "idLattes") %>% }
\CommentTok{#   left_join(}
\CommentTok{#     unb.prof.df.areas.de.atuacao %>% }
\CommentTok{#       select(area, idLattes) %>% }
\CommentTok{#       group_by(idLattes) %>% }
\CommentTok{#       summarise(n_distinct(area)), }
\CommentTok{#     by = "idLattes") %>% }
\CommentTok{#   left_join(}
\CommentTok{#     unb.prof.df.capes %>% }
\CommentTok{#       select(AreaPos, idLattes) %>% }
\CommentTok{#       group_by(idLattes) %>% }
\CommentTok{#       summarise(n_distinct(AreaPos)), }
\CommentTok{#     by = "idLattes")}

\CommentTok{# glimpse(unb.prof.df)}
\end{Highlighting}
\end{Shaded}

\subsection{CRISP-DM Fase.Atividade 3.4 - Integração dos
dados}\label{crisp-dm-fase.atividade-3.4---integraaao-dos-dados}

Como já informado, a integração dos dados envolve a união (merge) de
diferentes tabelas para criar um único \emph{dataset} para ser
utilizado no R, por exemplo.

\subsection{CRISP-DM Fase.Atividade 3.5 - Formatação dos
dados}\label{crisp-dm-fase.atividade-3.5---formataaao-dos-dados}

Como já informado, a formatação de dados envolve a realização de
pequenas alterações na estrutura dos dados, como a ordem das
variáveis, para permitir a execução de determinado método de data
mining.

\section{\texorpdfstring{CRISP-DM Fase 4 -
\textbf{Modelagem}}{CRISP-DM Fase 4 - Modelagem}}\label{crisp-dm-fase-4---modelagem}

Como já informado, na fase de \textbf{Modelagem} no CRISP-DM ocorre a
construção e avaliação de modelos estatísticos ou computacionais,
podendo ser realizada em quatro atividades genéricas, a seguir
organizadas na forma de seções

\subsection{CRISP-DM Fase.Atividade 4.1 - Seleção das técnicas de
modelagem}\label{crisp-dm-fase.atividade-4.1---seleaao-das-tacnicas-de-modelagem}

\subsection{CRISP-DM Fase.Atividade 4.2 - Realização de testes de
modelagem}\label{crisp-dm-fase.atividade-4.2---realizaaao-de-testes-de-modelagem}

Como já informado, na realização de testes de modelagem diferentes
modelos estatísticos ou computacionais são previamente testados e
avaliados. Pode-se dividir o \emph{dataset} criado na etapa anterior
para se ter uma base de treino na construção de modelos, e outra
pequena parte para validar e avaliar a eficiência de cada modelo criado
até se chegar ao mais “eficiente”.

\subsection{CRISP-DM Fase.Atividade 4.3 - Construção do modelo
definitivo}\label{crisp-dm-fase.atividade-4.3---construaao-do-modelo-definitivo}

Como já informado, a construçao do modelo definitivo é realizada com
base na melhor experiência do passo anterior.

\subsection{CRISP-DM Fase.Atividade 4.4 - Avaliação do
modelo}\label{crisp-dm-fase.atividade-4.4---avaliaaao-do-modelo}

\section{\texorpdfstring{CRISP-DM Fase 5 -
\textbf{Avaliação}}{CRISP-DM Fase 5 - Avaliação}}\label{crisp-dm-fase-5---avaliaaao}

Como já informado, na fase de \textbf{Avaliação} do CRISP-DM os
resultados não são apenas avaliados, mas se verifica se existem
questões relacionadas Ã~ organização que não foram suficientemente
abordadas. Deve-se refletir se o uso arepetido do modelo criado pode
trazer algum “efeito colateral” para a organização.

Como já informado, nesta fase, pode-se trabalhar com 3 atividades
genéricas, a seguir distribuídas em seções.

\subsection{CRISP-DM Fase.Atividade 5.1 - Avaliação dos
resultados}\label{crisp-dm-fase.atividade-5.1---avaliaaao-dos-resultados}

\subsection{CRISP-DM Fase.Atividade 5.2 - Revisão do
processo}\label{crisp-dm-fase.atividade-5.2---revisao-do-processo}

Como já informado, durante a revisão do processo verifica-se se o
modelo foi construído adequadamente. As variáveis (passadas) para
construir o modelo estarão disponíveis no futuro?

\subsection{CRISP-DM Fase.Atividade 5.3 - Determinação dos etapas
seguintes}\label{crisp-dm-fase.atividade-5.3---determinaaao-dos-etapas-seguintes}

Como já informado, pode ser necessário decidir-se por finalizar o
projeto, passar Ã~ etapa de desenvolvimento, ou rever algumas fases
anteriores para a melhoria do projeto.

\section{\texorpdfstring{CRISP-DM Fase 6 - \textbf{Implantação}
(\emph{deployment})}{CRISP-DM Fase 6 - Implantação (deployment)}}\label{crisp-dm-fase-6---implantaaao-deployment}

Como já informado, na fase de \textbf{Implantação}
(\emph{deployment}) se realiza o planejamento de implantação dos
produtos desenvolvidos (scripts, no caso do executado nesta disciplina)
para o ambiente operacional, para seu uso repetitivo, envolvendo
atividades de monitoramento e manutenção do sistema (script)
desenvolvido. A fase de implantação concluir com a produção e
apresentação do relatório final com os resultados do projeto.

Como já informado, são seis as atividades genéricas na fase de
\textbf{implantação}, a seguir apresentadas na forma de seções.

\subsection{CRISP-DM Fase.Atividade 6.1 - Planejamento da
transição}\label{crisp-dm-fase.atividade-6.1---planejamento-da-transiaao}

De que forma os produtos desenvolvidos pelo grupo poderiam ser colocados
em uso prático regular, na organização cliente?

\subsection{CRISP-DM Fase.Atividade 6.2 - Planejamento do monitoramento
dos
produtos}\label{crisp-dm-fase.atividade-6.2---planejamento-do-monitoramento-dos-produtos}

De que forma seria possível realizar o monitoramento do funcionamento
dos produtos em utilização no ambiente operacional?

\subsection{CRISP-DM Fase.Atividade 6.3 - Planejamento de
manuteção}\label{crisp-dm-fase.atividade-6.3---planejamento-de-manuteaao}

que manutenções, ajustes, mudanças, poderia ter que ser eventualmente
realizadas no produto (scripts), quando em uso no ambiente operacional
do cliente?

\subsection{CRISP-DM Fase.Atividade 6.4 - Produção do relatório
final}\label{crisp-dm-fase.atividade-6.4---produaao-do-relatario-final}

A entrega do relatório do grupo, tomando como base este aqui, reflete a
execução desta etapa.

\subsection{CRISP-DM Fase.Atividade 6.5 - Apresentação do relatório
final}\label{crisp-dm-fase.atividade-6.5---apresentaaao-do-relatario-final}

Como já informado, não será feita apresentação do relatório, mas
esperamos que publicações científicas possam ser geradas com pelo seu
grupo, com o apoio dos professores da disciplina.

\subsection{CRISP-DM Fase.Atividade 6.6 - Revisão sobre a execução do
projeto}\label{crisp-dm-fase.atividade-6.6---revisao-sobre-a-execuaao-do-projeto}

Deve-se fazer aqui o registro de lições aprendidas, bem como traçadas
perspectivas futuras de aprimoramento deste trabalho, da disciplina de
Ciência de Dados para Todos etc.

\section{Referências}\label{referancias}

\begin{itemize}
\tightlist
\item
  Azevedo, Mário Luiz Neves de, João Ferreira de Oliveira, e Afrânio
  Mendes Catani. “O Sistema Nacional de Pós-Graduação (SNPG) e o
  Plano Nacional de Educação (PNE 2014-2024): regulação, avaliação
  e financiamento”. Revista Brasileira de Política e Administração
  da Educação 32, nº 3 (2016).
  \url{http://dx.doi.org/10.21573/vol32n32016.68576}.
\item
  Can, Fazli, Tansel Özyer, e Faruk Polat, orgs. State of the Art
  Applications of Social Network Analysis. Lecture Notes in Social
  Networks. Switzerland: Springer International Publishing, 2014.
\item
  CAPES. “Documentos de Área”. CAPES.gov.br. Acessado 12 de junho
  de 2018.
  \url{http://avaliacaoquadrienal.capes.gov.br/documentos-de-area}.
\item
  ———. “Plano Nacional de Pós-Graduação - PNPG 2011/2020 Vol.
  1”. Brasília - DF, dezembro de 2010.
  \url{http://www.capes.gov.br/images/stories/download/Livros-PNPG-Volume-I-Mont.pdf}.
\item
  ———. “Plano Nacional de Pós-Graduação - PNPG 2011/2020 Vol.
  2”. Brasília - DF, dezembro de 2010.
  \url{http://www.capes.gov.br/images/stories/download/PNPG_Miolo_V2.pdf}.
\item
  ———. “Sucupira: coleta de dados, docentes de pós-graduação
  stricto sensu no Brasil 2015”. CAPES - Banco de Metadados, 16 de
  março de 2016.
  \url{http://metadados.capes.gov.br/index.php/catalog/63}.
\item
  Chapman, Pete, Julian Clinton, Randy Kerber, Thomas Khabaza, Thomas
  Reinartz, Colin Shearer, e Rüdiger Wirth. “CRISP-DM 1.0:
  Step-by-Step Data Mining Guide”. USA: CRISP-DM Consortium, 2000.
  \url{https://www.the-modeling-agency.com/crisp-dm.pdf}.
\item
  Datacamp. “Machine Learning with R (Skill Track)”. Datacamp, 2018.
  \url{https://www.datacamp.com/tracks/machine-learning}.
\item
  Fernandes, Jorge H C, e Ricardo Barros Sampaio.
  “DataScienceForAll”. Zotero, 13 de junho de 2018.
  \url{https://www.zotero.org/groups/2197167/datascienceforall}.
\item
  ———. “Especificação do Trabalho Final da Disciplina de
  Ciência de Dados para Todos 2017.2: Estudo sobre a visibilidade
  internacional da produção científica das pós-graduações
  vinculadas Ã~s áreas de conhecimento da CAPES, na Universidade de
  Brasília (Comunicação Interna)”. Disciplina 116297 - Tópicos
  Avançados em Computadores, turma D, do semestre 2017.2, do
  Departamento de Ciência da Computação do Instituto de Ciências
  Exatas da Universidade de Brasília, 28 de novembro de 2017.
  \url{https://aprender.ead.unb.br/pluginfile.php/474549/mod_resource/content/1/Estudo\%20da\%20Cie\%CC\%82ncia.pdf}.
\item
  Fernandes, Jorge H C, Ricardo Barros Sampaio, e João Ribas de Moura.
  “Ciência de Dados para Todos (Data Science For All) - 2018.1 -
  Análise da Produção Científica e Acadêmica da Universidade de
  Brasília - Modelo de Relatório Final da Disciplina - Departamento de
  Ciência da Computação da UnB”. Disciplina 116297 - Tópicos
  Avançados em Computadores, turma D, do semestre 2018.1, do
  Departamento de Ciência da Computação do Instituto de Ciências
  Exatas da Universidade de Brasília, 13 de junho de 2018.
\item
  Frickel, Scott, e Kelly Moore. The New Political Sociology of Science:
  Institutions, Networks, and Power. Science and technology in society.
  USA: The University of Wisconsin Press, 2006.
\item
  Graduate Prospects Ltd. “Job profile: Higher education lecturer”,
  2018.
  \url{https://www.prospects.ac.uk/job-profiles/higher-education-lecturer}.
\item
  Kalpazidou Schmidt, Evanthia, e Ebbe Krogh Graversen. “Persistent
  factors facilitating excellence in research environments”. Higher
  Education 75, nº 2 (1º de fevereiro de 2018): 341â€``63.
  \url{https://doi.org/10.1007/s10734-017-0142-0}.
\item
  Kilduff, Martin, e Wenpin Tsai. Social Networks and Organizations. UK:
  Sage Publications, 2003.
\item
  Kolaczyk, Eric D., e Gábor Csárdi. Statistical Analysis of Network
  Data with R. USA: Springer, 2014.
\item
  Kuhn, Max, Jed Wing, Steve Weston, Andre Williams, Chris Keefer, Allan
  Engelhardt, Tony Cooper, et al. “Package ‘Caret’ -
  Classification and Regression Training”, 27 de maio de 2018.
  \url{https://cran.r-project.org/web/packages/caret/caret.pdf}.
\item
  Leite, Fernando César Lima. “Considerações básicas sobre a
  Avaliação do Sistema Nacional de Pós-Graduação”. Comunicação
  Pessoal (slides). Universidade de Brasília, abril de 2018.
  \url{https://aprender.ead.unb.br/pluginfile.php/502250/mod_resource/content/1/Considera\%C3\%A7\%C3\%B5es\%20b\%C3\%A1sicas\%20sobre\%20a\%20Avalia\%C3\%A7\%C3\%A3o\%20do\%20Sistema\%20Nacional.pdf}.
\item
  Lusher, Dean, Johan Koskinen, e Garry Robins, orgs. Exponential Random
  Graph Models for Social Networks: Theory, methods, and applications.
  Structural Analysis in the Social Sciences. USA: Cambridge University
  Press, 2013.
\item
  Mariscal, Gonzalo, Ã``scar Marbán, e Covadonga Fernández. “A
  survey of data mining and knowledge discovery process models and
  methodologies”. The Knowledge Engineering Review 25, nº 2 (2010):
  137â€``66. \url{https://doi.org/10.1017/S0269888910000032}.
\item
  Nery, Guilherme, Ana Paula Bragaglia, Flávia Clemente, e Suzana
  Barbosa. “Nem tudo parece o que é: Entenda o que é plágio”.
  Instituto de Arte e Comunicação Social da UFF, 2009.
  \url{http://www.noticias.uff.br/arquivos/cartilha-sobre-plagio-academico.pdf}.
\item
  Nooy, Wouter de, Andrej Mrvar, e Vladimir Batagelj. Exploratory Social
  Network Analysis with Pajek. Structural Analysis in the Social
  Sciences. USA: Routldge, 2005.
\item
  Pátaro, Cristina Saitê de Oliveira, e Frank Antonio Mezzomo.
  “Sistema Nacional de Pós-Graduação no Brasil: estrutura,
  resultados e desafios para política de Estado - Lívio Amaral”.
  Revista Educação e Linguagens 2, nº 3 (julho de 2013): 11â€``17.
\item
  Schwartzman, Simon. “A Ciência da Ciência”. Ciência Hoje 2, nº
  11 (março de 1984): 54â€``59.
\item
  Silver, Nate. The Signal and the Noise: Why so many predictions fail
  — but some don’t. USA: The Penguin Press HC, 2012.
\item
  Vicari, Donatella, Akinori Okada, Giancarlo Ragozini, e Claus Wiehs.
  Analysis and Modeling of Complex Data in Behavioral and Social
  Sciences. Studies in Classifi cation, Data Analysis, and Knowledge
  Organization. Switzerland: Springer, 2014.
\item
  Wickham, Hadley, e Garrett Grolemund. R for Data Science: Import,
  Tidy, Transform, Visualize, and Model Data. USA: O’Reilly, 2016.
\end{itemize}


\end{document}
